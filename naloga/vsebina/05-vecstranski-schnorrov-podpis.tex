% LTeX: language=sl-SI
\section{Večstranski Schnorrov podpis}
\label{sec:multischnorr}
Micali et al.~\cite{micali2001asm} so prvi definirali formalni model za večstranske podpise in podali
formalni dokaz varnosti za podpisno shemo v tem modelu. Zamislili so si formalni model za večstranske
podpise in ga poimenovali \textbf{večstranski podpis podskupine z odgovornostjo} (angl.\
\textit{Accountable-Subgroup Multisignature (ASM)}). V tem razdelku predstavimo njihov model, primer
večstranskega podpisa, ki temelji na Schnorrovem podpisu, in argumentiramo varnost podpisa oz.\
modela na sploh.

\subsection{Večstranski podpis podskupine z odgovornostjo}
Kljub daljšemu imenu, večstranski podpis podskupine z odgovornostjo le predstavlja formalni model za
večstranske pospise, predstavljene v razdelku~\ref{sec:multisig}. Ideja oz.\ cilj večstranskega
podpisa je torej, da lahko katerakoli podskupina podpisnikov $S$, neke skupine $P$, brez potrebe po
\textbf{centru zaupanja} (angl.\ \textit{trusted third party (TTP)}) ustvari podpis. Generiranje
ključev je torej popolnoma v domeni skupine $P$. Podpis, ki ga ustvari $S$ predstavlja splošno
preverljiv dokaz strukture $S$ in dejstva, da vsak član skupine stoji za (torej podpisuje) sporočilom
$m$. V tem razdelku predstavimo osnovno idejo ASM.

\begin{definicija}[Večstranski podpis podskupine z odgovornostjo]
\label{def:asm}
    Skupina $P$ je sestavljena iz $L$ podpisnikov, torej
    $$
    P = P_1, P_2, \dots, P_L.
    $$
    Podpisnik predstavlja naključnostni polinomski algoritem. Vsak podpisnik pozna svojo 
    identifikacijsko številko (eno od števil $1, \dots, L$) in pa \textit{varnostni parameter} $k$,
    ki je enak za vse podpisnike.

    Kot vsi digitalni podpisi iz definicije~\ref{def:digisig}, ima tudi ta štiri glavne komponente:
    \begin{itemize}
        \item $\mathcal{P}$ je algoritem za ustvarjanje parametrov. Pognan je samo enkrat, na začetku
            sodelovanja. Vrne javne parametre sheme $\text{Par}$. Velikost parametrov je odvisna od
            varnostnega parametra $k$. Velja torej
            $$
            \mathcal{P}(k) = \text{Par}.
            $$
        \item $\mathcal{G}$ je algoritem za ustvarjanje ključev. Za neko skupino podpisnikov $P$ je pognan samo
            enkrat, na začetku sodelovanja. Vsak podpisnik $P_i$ dobi kot vhod seznam vseh podpisnikov v $P$
            in parametre podpisne sheme $\text{Par}$ ter požene algoritem $\mathcal{G}$, ki vrne par ključev
            $(\text{pk}_i, \text{sk}_i)$. Zapišemo lahko torej
            $$
            \mathcal{G}_i(L, \text{Par}) = (\text{pk}_i, \text{sk}_i),
            $$
            kjer smo brez škode za splošnost predpostavili, da je vsak podpisnik enolično predstavljen
            s svojim indeksom $i \in \{1, 2, \dots, L\}$ v skupini $P$.
        \item $\mathcal{S}$ je protokol za podpisovanje. Pognan je vsakič, ko neka podskupina $S$ želi
            ustvariti podpis. Vsak podpisnik požene algoritem $\mathcal{S}$ s seznamom vseh podpisnikov
            $S$, njihovimi javnimi ključi $\text{pk}_j$, kjer $j$ teče po identifikacijskih številkah
            vseh članov $S$, sporočilom $m$ in lastnim zasebnim ključem $\text{sk}_i$ kot vhodnimi
            podatki. Algoritem $\mathcal{S}$ je porazdeljen protokol, pri izvedbi morajo sodelovati
            vsi člani $S$, ki med seboj komunicirajo. Po uspešni izvedbi lahko en od članov objavi
            podpis $\sigma$.
        \item $\mathcal{V}$ je algoritem za preverjanje veljavnosti podpisa. Požene ga tisti, ki želi 
            preveriti veljavnost večstranskega podpisa. Ni nujno, da je to eden izmed podpisnikov iz $P$. 
            Kot vhod algoritem dobi seznam podpisnikov $S$, pripadajoče javne ključe, sporočilo $m$ in 
            morebiten podpis $\sigma$. Algoritem potem vrne
            $$ 
            \mathcal{V}((id_1, \dots, id_l), (\text{pk}_{id_1}, \dots, \text{pk}_{id_l}), m, \sigma) = 
            \begin{cases}
                veljaven, & \text{podpis veljaven}, \\
                neveljaven, & \text{sicer}.
            \end{cases}
            $$
    \end{itemize}
\end{definicija}

\subsubsection{Robustnost, varnost in napadalec}
Večstranski podpis iz definicije~\ref{def:asm} ni \textit{robusten}. To pomeni, da v primeru izpada
enega od podpisnikov $P_i$, ki je del podskupine $S$, ta ne more ustvariti večstranskega podpisa.
Podpis še vedno lahko ustvari $S \setminus \{P_i\}$. Ker pri končni shemi ne iščemo robustnosti,
lahko definiramo zelo močnega napadalca, ki ima velik vpliv na celoten sistem.
\begin{definicija}[Napadalec pri ASM]
\label{def:asm-napadalec}
    Napadalca v modelu večstranskih podpisov podskupine z odgovornostjo bomo označili z $F$. Ima
    naslednje zmožnosti:
    \begin{itemize}
        \item Ima popoln nadzor nad vsemi komunikacijskimi kanali med člani skupine $P$. Lahko bere,
            spreminja in preprečuje dostavo vseh sporočil. Prav tako lahko v imenu kateregakoli
            podpisnika pošlje sporočilo.
        \item Kadarkoli lahko \textit{pokvari} kateregakoli podpisnika. Ko je podpisnik pokvarjen,
            napadalec izve njegovo celotno notranje stanje, vključno z vsemi skrivnostmi.
        \item Nadzira lahko vhod algoritma za ustvarjanje ključev $\mathcal{G}$ za vse podpisnike.
            Vsakemu lahko poda drugačno skupino $P$.
        \item Od kateregakoli nepokvarjenega podpisnika lahko kadarkoli zahteva podpis nekega sporočila
            skupaj s podskupino, ki jo določi napadalec. To sposobnost imenujemo
            \textit{napad z izbranim sporočilom in podskupino}.
    \end{itemize}
\end{definicija}

Zaradi obširnih zmožnosti napadalca, ta lahko vedno prepreči podpis sporočila. Naš cilj, kar se
tiče varnosti, je, da preprečimo eksistencialno ponarejanje podpisa. Želimo torej, da napadalec ni
zmožen ponarediti podpisa za katerokoli sporočilo v imenu katerekoli podskupine.

\begin{definicija}[Varnost pri ASM]
\label{def:asm-varnost}
    Naj bo $k$ varnostni parameter (ki si ga delijo vsi podpisniki). Naj bo $c > 0$ poljubna konstanta. 
    Naj bo $F$ napadalec, ki ustreza definiciji~\ref{def:asm-napadalec}, torej lahko nadzoruje in
    spreminja dejanja podpisnikov skupine $S$. Naj bo njegova računska moč omejena s polinomskim časom
    v parametru $k$. Naj bo $p$ verjetnost, da napadalec $F$ vrne trojico $(\sigma, m, S)$, za katero velja: 
    \begin{itemize}
        \item $\sigma$ je veljaven večstranski podpis sporočila $m$ s strani skupine $S$.
        \item Obstaja nepokvarjen podpisnik $P_j$ iz skupine $S$, od katerega napadalec $F$ ni zahteval podpisa 
            sporočila $m$ s strani skupine $S$.
    \end{itemize}
    Shemi za večstranske 
    podpise podskupine z odgovornostjo bomo rekli, da je \textbf{varna}, če je verjetnost $p$ zanemarljiva,
    oz.\ če za vsakega polinomsko omejenega napadalca $F$ obstaja zanemarljiva funkcija $\varepsilon$,
    da velja
    $$ 
    p < \varepsilon(k).
    $$
\end{definicija}

Kot je standardno pri varnostni obravnavi digitalnih podpisov, tudi tu predpostavimo, da preverjevalec
kljub močnemu napadalcu lahko vedno pridobi prave javne ključe podpisnikov iz $S$. To pomeni, da
v času preverjanja pozna prave identitete podpisnikov, tudi če napadalec lahko doseže, da podpisniki
v času podpisovanja ne vedo zares, s kom podpisujejo (ne morejo zaupati, da res sodelujejo s člani
podskupine $S$, saj napadalec nadzira komunikacijo).

\subsubsection{Slučajni orakelj pri ASM}
Varnostna obravnava shem za večstranske podpise podskupine z odgovornostjo zahteva model slučajnega
oraklja. Zato predpostavimo, da je $k_2$ še en varnostni parameter. Vsi člani skupine $P$ in
napadalec imajo dostop do slučajnega oraklja $H: \{0, 1\}^* \rightarrow \{0, 1\}^{k_2}$, ki je
naključno izbrana funkcija med vsemi funkcijami, ki slikajo med $\{0, 1\}^*$ in $\{0, 1\}^{k_2}$.

\subsection{Konstrukcija večstranskega Schnorrovega podpisa}
V nadaljevanju bomo konstruirali večstransko verzijo Schnorrovega podpisa, predstavljenega v
poglavju~\ref{sec:schnorr}. Končna shema spada med večstranske podpise podskupin z odgovornostjo.
Ideja konstrukcije je, da začnemo z naivno verzijo, nato pa rešimo njene probleme, kar nas
privede do formalne definicije v razdelku~\ref{sec:def_multi_schnorr} in dokaza varnosti v
razdelku~\ref{sec:proof_multi_schnorr}.

\subsubsection{Naivna verzija}
Vsi podpisniki v skupini $P$ se strinjajo o izboru končne grupe $G$ reda $q$, ki jo generira generator
$g$. Vsak podpisnik $P_i$ si neodvisno in naključno izbere skrivno število $s_i \in \Z_q$ in izračuna 
$$
I_i = g^{s_i}.
$$
Tako vsak podpisnik ustvari svoj par ključev
\begin{align*}
    \text{pk}_i &= (q, g, I_i), \\
    \text{sk}_i &= s_i.
\end{align*}

Poljubna podskupina $S =\{P_{id_1}, \dots, P_{id_l}\}$ skupine $P$ podpiše sporočilo $m$ s tremi 
krogi komunikacije:
\begin{enumerate}
    \item Vsak podpisnik $P_i$ iz podskupine $S$ si izbere naključen element $r_i \in \Z_q$
        in izračuna zavezo
        $$
        X_i = g^{r_i}.
        $$
        Podpisniki potem pošljejo svoje zaveze izbranemu podpisniku $D$. 
    \item $D$ izračuna \textit{skupno zavezo} 
        $$ 
        \tilde{X} = X_{id_1} \cdot X_{id_2} \cdot \cdots \cdot X_{id_l}.
        $$
        in jo pošlje vsem podpisnikom.
    \item Vsak podpisnik s pomočjo slučajnega oraklja izračuna izziv 
        $$ 
        e = H(\text{enc}(\tilde{X}) || m || S)
        $$
        in svoje \textit{individualne podpise} 
        $$ 
        y_{id_i} = e s_{id_i} + r_{id_i} \bmod q.
        $$
        Individualni podpisi so potem spet poslani podpisniku $D$, ta pa sedaj lahko 
        izračuna 
        $$ 
        \tilde{y} = (y_{id_1} + y_{id_2} + \cdots + y_{id_l}) \bmod q 
        $$
        in vrne končen večstranski podpis
        $$ 
        \sigma = (\tilde{X}, \tilde{y}).
        $$
\end{enumerate}

Preverjanje veljavnosti podpisa je podobno kot pri navadnemu Schnorrovemu podpisu. Da preverimo 
podpis $(\tilde{X}', \tilde{y}')$ sporočila $m$ najprej izračunamo 
$$
e' = H(\text{enc}(\tilde{X}') || m || S)
$$
in preverimo, če velja
\begin{equation}
\label{eq:naive-ver}
g^{\tilde{y}'} \stackrel{?}{=} \tilde{X}' \cdot \left(\prod_{P_i \in S} I_i \right)^{e'}.
\end{equation}

\subsubsection{Generiranje parametrov in Predpostavka DL}
Micali et al.~\cite{micali2001asm} so izpostavili in rešili več problemov z zgornjo naivno idejo.
Prvi problem se pojavi pri generiranju skupnih parametrov javnega ključa. Ker si podpisniki
delijo samo varnostni parameter $k$ in pa slučajnega oraklja $H$, morajo za parametre uporabiti
$H$ na vnaprej dogovorjen način. Napadalec bo tako poznal točen postopek ustvarjanja skupnih
parametrov, kar mu potencialno da prednost. Izkaže se, da lahko s primerno uporabo $H$ poskrbimo,
da to ne predstavlja nevarnosti.

Ker pa je pri uporabi kriptografije standardno, da se uporabi znane, vnaprej definirane grupe, se
s problemom generiranja parametrov ne bomo ukvarjali. Enostavno predpostavimo, da imajo podpisniki
dostop do skupnih, varnih parametrov.

Ker podpis temelji na Schnorrovem, varnost temelji na težavnosti diskretnega logaritma. Predpostavka
o varnosti je tako standardna predpostavka o težavnosti diskretnega logaritma, predstavljena v
definiciji~\ref{def:dl}. Pri obravnavi sheme torej predpostavimo, da je $G$ grupa, v kateri je
izračun diskretnega logaritma težaven.

\begin{primer}[Generianje parametrov v grupi $\Z_p^*$]
    V najosnovnejši obliki podpisa uporabimo grupo $\Z_p^*$, kjer je $p$ praštevilo. Za delovanje
    potrebujemo še praštevilo $q$, ki deli $p - 1$. Za generiranje lahko uporabimo standarden algoritem
    za pridobivanje praštevil~\ref{alg:gen}.
    \begin{algorithm}
        \caption{Algoritem $GenPrimes(k)$ za generiranje praštevil.}
        \label{alg:gen}
        \begin{algorithmic}
            \State $q \gets \text{naključno izbrano $k$-bitno število}$
            \State $p \gets 2q + 1$
            \While{$q$ ni praštevilo in $p$ ni praštevilo}
                \State $q \gets \text{naključno izbrano $k$-bitno število}$
                \State $p \gets 2q + 1$
            \EndWhile
            \State \Return{$p, q$}
        \end{algorithmic}
    \end{algorithm}

    Kot primer predstavimo še točno predpostavko o težavnosti diskretnega logaritma v grupi $\Z_p^*$.
    \begin{definicija}[Predpostavka o težavnosti diskretnega logaritma pri ASM v grupi $\Z_p^*$]
    \label{def:asm_dlp}
        Naj bo $A$ poljuben naključnostni algoritem, ki teče v polinomskem času in svoje parametre sprejme 
        \begin{itemize}
            \item praštevili $p$ in $q$, da velja $p = 2q + 1$ in $q$ je dolg $k$-bitov,
            \item naključni element $g \in \Z_p^*$ reda $q$,
            \item naključni element $I$ iz podgrupe $\Z_p^*$, ki jo generira $g$.
        \end{itemize}
        Naj $p_k^A$ označuje verjetnost, da $A$ vrne $s \in \Z_q$, tako da velja
        $$
        I \equiv g^s \pmod p.
        $$
        Potem za vsako konstanto $c > 0$ in za vsak dovolj velik $k$ velja, da je verjetnost $p_k^A$
        zanemarljiva, oz.\ da obstaja zanemarljiva funkcija $\varepsilon$, za katero velja
        $$ 
        p_k^A < \varepsilon(k).
        $$
    \end{definicija}

\end{primer}

\subsubsection{Napad na generiranje ključev}
Naslednja težava nastopi v obliki napada v fazi generiranja ključev. Ker napadalec nadzoruje vso
komunikacijo, lahko v tej fazi enostavno počaka, da vsi podpisniki ustvarijo svoje ključe. Na tej
točki napadalec stopi v vlogo (torej pokvari) zadnjega podpisnika $P_L$ in generira nov par ključev.
Če si torej izbere naključen $s \in \Z_q$, lahko izračuna svoj javni ključ kot
$$
I_L = \left( \prod_{i=1}^{L-1} I_i \right)^{-1} \cdot g^s.
$$
To mu omogoča, da se podpisuje v imenu celotne podskupine $S$, saj se bodo pri množenju v
enačbi~\eqref{eq:naive-ver} za preverjanje tako izničil vsi ostali javni ključi:
\begin{align*}
    g^{\tilde{y}} &\stackrel{?}{=} \tilde{X} \cdot \left(\prod_{i=1}^L I_i \right)^{e}. \\
                   &\stackrel{?}{=} \tilde{X} \cdot \left(\left(\prod_{i=1}^{L-1} I_i \right) \cdot I_L \right)^{e}. \\
                   &\stackrel{?}{=} \tilde{X} \cdot \left(\left(\prod_{i=1}^{L-1} I_i \right) \cdot \left( \prod_{i=1}^{L-1} I_i \right)^{-1} \cdot g^s \right)^{e}. \\
                   &\stackrel{?}{=} \tilde{X} \cdot \left( g^s \right)^{e}. \\
\end{align*}
To pomeni, da na veljavnost podpisa vpliva samo napadalec, torej lahko ponaredi katerikoli podpis.

Za rešitev tega problema lahko uporabimo na dokaze o znanju brez razkritja znanja. Torej, da
preprečimo napad na generiranje ključev, od vsakega podpisnika $P_i$ zahtevamo, da poleg svojega
javnega ključa $I_i$ objavi tudi dokaz o znanju brez razkritja znanja za svoj zasebni ključ glede na
javni ključ. Objaviti mora torej dokaz, da pozna diskretni logaritem $I_i$ z bazo $g$. Ta dokaz je
neinteraktiven, saj obravnavmo model slučajnega oraklja in lahko uporabimo Fiat-Shamirjevo
hevristiko~\ref{sec:fiat-shamir}.

Ker tovrstni dokazi brez preverjevalca nimajo smisla, se je na tej točki vredno vprašati, kdo bo preverjal
veljavnost dokazov. Izkaže se, da je najbolj enostavno, da se dokaze doda v posamezne javne ključe.
Tako pade breme preverjanja na tistega, ki bo preverjal podpis. Problematično je dejstvo, da taka
sprememba podaljša javne ključe in privede do izgube učinkovitosti. Preverjanje veljavnosti podpisa
bo sedaj poleg dveh potenciranj, ki sta v enačbi~\eqref{eq:gen-schnorr-ver} potrebni za preverjanje
standarnega Schnorrovega podpisa, potrebovalo še dodatnih $2|S|$ potenciranj, saj je potrebno
preveriti vse dokaze javnih ključev, kar ustreza preverjanju $|S|$ Schnorrovih podpisov. Posamezni
preverjevalec podpisov lahko sicer dodatno delo opravi le enkrat za vsako skupino, če natančno beleži
rezultate preverjanja dokazov.

\subsubsection{Učinkovitost podpisovanja: Merklova drevesa}
Kot bomo videli, je za dokaz varnosti potrebno, da lahko algoritem $A$, ki bo simuliral
proces podpisovanja napadalcu, v polinomskem času za vsakega pokvarjenega podpisnika $P_j$ pridobi
diskretni logaritem $I_j$ iz predloženega dokaza o znanju brez razkritja znanja. Izkaže pa se, da če
bo $P_j$ dokaz izračunal s $q$ klici slučajnega oraklja, potem bo algoritem $A$ uspešno
pridobil diskretni logaritem z verjetnostjo največ $1/q$. Še več, če je $b$ pokvarjenih podpisnikov,
bo algoritem $A$ uspešen z verjetnostjo največ $1/q^b$. Z drugimi besedami, če želimo,
da algoritem $A$ teče v polinomskem času, je podpisnikov lahko največ logaritemsko mnogo v
številu klicev oraklja $q$. Ker pa je teh klicev lahko največ polinomsko mnogo v varnostnem parametru
$k$, je torej število podpisnikov lahko največ logaritemsko v $k$.

Ta problem lahko rešimo tako, da vsak podpisnik $P_i$ najprej izračuna svoj par ključev $(s_i, I_i)$
in na podlagi naključno izbrane vrednosti $r_i$ tudi zavezo $X_i = g^{r_i} \pmod p$, ki bi jo uporabil
pri dokazu znanja brez razkritja znanja. Potem si vsi podpisniki izmenjajo javne ključe $I_i$ in
zaveze $X_i$. Vsak podpisnik lahko sedaj izračuna \textit{skupni izziv}
$$
e = H(\text{enc}(X_1) || \text{enc}(I_1) || \text{enc}(X_2) || \text{enc}(I_2) || \dots || \text{enc}(X_L) || \text{enc}(I_L)).
$$
Podpisniki nato dokončajo svoj dokaz na podlagi zgostitve $e$ in svojega para ključev.

Kot bomo videli, bo to pomenilo, da za generiranje dokazov $b$ pokvarjenih podpisnikov sedaj rabi le
$bq$ klicev oraklja, torej je polinomski algoritem $A$ uspešen z verjetnostjo $1/bq$. To
omogoča, da je podpisnikov več kot logaritemsko mnogo. Pojavi pa se drugačen problem z učinkovitostjo,
saj mora sedaj vsak podpisnik v svojem javnem ključu hraniti še Schnorrov podpis sporočila $e$ in pa
tudi vse javne ključe sopodpisnikov in njihove zaveze. Brez teh podatkov namreč dokaz znanja (Schnorrov
podpis) ni preverljiv. To pomeni, da je za katerokoli podskupino $S$ velikost ključa proporcionalna
velikosti celotne skupine $P$. Poleg tega mora sedaj preverjevalec preveriti, da se vsi seznami javnih
ključev ujemajo. V primerjavi z osnovno naivno idejo, kjer je preverjevalec potreboval le $|S|$ navadnih
Schnorrovih javnih ključev, je v tej izvedbi velikost javnega ključa nedopustno velika.

Da nazaj pridobimo učinkovit podpis, lahko uporabimo slučajnega oraklja in kriptografsko
orodje, imenovano \textbf{Merklovo drevo} (angl.\ \textit{Merkle tree}). V osnovi je to
\textit{binarno drevo}, torej drevo, kjer ima vsako vozlišče največ dva otroka. Ideja je, da v liste
shranimo zgostitve katerekolih podatkov, pridobljene s pomočjo izbranega slučajnega oraklja ali zgoščevalne
funkcije. Drevo potem gradimo navzgor tako, da v vsako vozlišče shranimo zgostitev stika levega in desnega
otroka. Tako nadaljujemo do korena. Pomembno je, da je za pridobivanje zgostitev uporabljena zgoščevalna
funkcija, za katero je računsko neizvedljivo najti trk, ali pa slučajni orakelj. Osnovna
struktura je prikazana na sliki~\ref{fig:merkle}.
\begin{figure}[ht]
  \centering
  \begin{tikzpicture}[<-, >=stealth', auto, semithick,
   level/.style={sibling distance=60mm/#1}]

  \node [rectangle,draw] {$H(H(H(x_1) || H(x_2)) || H(H(x_3) || H(x_4)))$}
    child {node [rectangle,draw] {$H(H(x_1) || H(x_2))$}
        child {node [rectangle,draw] {$H(x_1)$}
            child {node [rectangle,draw] {$x_1$}}
        }
        child {node [rectangle,draw] {$H(x_2)$}
            child {node [rectangle,draw] {$x_2$}}
        } 
    }
    child {node [rectangle,draw] {$H(H(x_3) || H(x_4))$}
        child {node [rectangle,draw] {$H(x_3)$}
            child {node [rectangle,draw] {$x_3$}}
        } 
        child {node [rectangle,draw] {$H(x_4)$}
            child {node [rectangle,draw] {$x_4$}}
        }
    };
  \end{tikzpicture}
  \caption[Merklovo drevo.]{Primer Merklovega drevesa s štirimi listi. Imamo torej $4$ vhodne podatke
    $x_1, x_2, \dots, x_4$, na vsakem nivoju drevesa se izračuna stik in zgostitev. Ustvarjeno s 
    pomočjo~\cite{SOtikz}.}
  \label{fig:merkle}
\end{figure}

Pomembno je opaziti, da če zgoščevalna funkcija ustvarja zgostitve, dolge $k$ bitov, bodo toliko
dolgi vsi podatki, ki jih hranijo vsa vozlišča drevesa. Za Merklova drevesa je ključno, da je v primeru
varne zgoščevalne funkcije računsko zelo težko spremeniti katerokoli vrednost v drevesu, brez da bi se
spremenila vrednost v korenu. Enako velja za spremembe vhodnih podatkov. Merklova drevesa nam torej
omogočajo, da se zavežemo $L$ vrednostim $x_1, x_2, \dots, x_L$ tako, da enostavno vrnemo en sam
$k$-bitni niz. To storimo tako, da shranimo zgostitve vrednosti $x_1, x_2, \dots, x_L$ v liste drevesa
in izračunamo koren. Vrednostim se zavežemo tako, da preverjevalcu damo vrednost korena. Ko mu kasneje
želimo razkriti vrednosti, mu jih enostavno povemo, on pa lahko z izračunom drevesa preveri, da so
vrednosti prave. Povzeto po~\cite{micali2000csproofs}.

\begin{trditev}[Overjevalna pot~\cite{micali2000csproofs}]
\label{trd:overjevalna_pot}
    Naj bo $A$ zavezovalec, ki se želi zavezati vrednostim $x_1, x_2, \dots, x_L$ preverjevalcu $B$. Če
    se $A$ zaveže tako, da $B$ pove vrednost korena Merklovega drevesa, izračunanega na podlagi
    vrednosti $x_1, x_2, \dots, x_L$, potem lahko dokaže zavezo eni vrednosti $x_i$ ($1 \le i \le L$)
    tako, da preverjevalcu $B$ pove $1 + \lceil \log L \rceil$ vrednosti: $x_i$ in pa \textbf{overjevalno
    pot}, torej vrednosti, shranjene v sestrskih vozliščih vseh vozlišč na poti med vključno $x_i$ in
    korenom (brez korena, ki ga $B$ že pozna).
\end{trditev}

\begin{dokaz}
    Trditev lahko dokažemo z indukcijo na globino Merklovega drevesa $d$. Naj bo $x$ vrednost, za katero
    želimo dokazati zavezo in $w_1, w_2, \dots, w_d$ overjevalna pot.
    \begin{itemize}
        \item $d = 2$: Če je globina drevesa $2$, želimo pokazati, da moramo poleg $x$ povemo samo eno
            dodatno vrednost. Ta vrednost je torej zgostitev $x$-ove sosednje vrednosti $y$. Ko $B$ izračuna
            $H(x || y)$, je to že koren, in lahko vrednost primerja z zabeleženo.
        \item $n \rightarrow n + 1$: Predpostavimo torej, da za globino drevesa $n$ lahko preverimo zavezo s
            podanimi $1 + d$ vrednostmi. To nam omogoča izračuna vrednosti v enem od korenovih otrok v
            drevesu globine $n + 1$, po indukcijski predpostavki. Če poznamo še vrednost drugega otroka, je
            možen izračun korena.
    \end{itemize}
\end{dokaz}

Varnost Merklovih dreves temelji na odpornosti zgoščevalne funkcije proti trkom. Ker je vsako vozlišče
drevesa zgostitev kombinacije otrok, se kakršnakoli sprememba v drevesu prenese v spremembo korena.
Torej javna objava korena zaveže objavitelja k vrednostim v listih. Da bi lahko spremenil vrednost
lista brez spremembe korena, bi moral najdi trk v zgoščevalni funkciji, in zamenjati vrednost lista
z vrednostjo, ki daje enako zgostitev. Če so uporabljene varne zgoščevalne funkcije, je to računsko
neizvedljivo.

Merklova drevesa lahko sedaj uporabimo, da naredimo večstranski podpis bolj učinkovit. Podpisniki
še vedno predložijo dokaze znanja prek Schnorrovega podpisa, po izmenjavi dokazov pa vsak izračuna
Merklovo drevo z listi $I_1, I_2, \dots, I_L$. To drevo bo torej globine $\log L$. Za svoj javni
ključ potem vsak podpisnik $P_i$ nastavi $I_i$ in overjevalno pot za $I_i$. Ker je ponavadi $I_i$
veliko daljši od dolžine zgostitev, in ker je v overjevalni poti le $\log L$ zgostitev, se ključ
ne podaljša veliko (npr.\, če je $I_i$ dolg $2000$ bitov in so zgostitve $200$-bitne, bo $1000$
podpisnikov ustvarilo ključe, dolge manj od $4000$ bitov, torej manj kot dvakrat daljše).

Ko sedaj preverjevalec preverja podpis sporočila $m$ s strani $S$, lahko za vsakega podpisnika
izračuna vrednost korena Merklovega drevesa in preveri, da koreni za vse podpisnike ujemajo. Obstoj
enega nepokvarjenega podpisnika tako prisili vse pokvarjene, da uporabljajo prave ključe (ali pa
najdejo trk v zgoščevalni funkciji, kar je izjemno težko). S to spremembo smo tako uspešno ohranili
podpise varne in časovno učinkovite.

\subsection{Definicija večstranskega Schnorrovega podpisa}
\label{sec:def_multi_schnorr}
Sedaj smo pripravili vse potrebno, da podamo natančno definicijo končne sheme.

\subsubsection{Skupni parametri}
Vsi podpisniki se strinjajo o skupnem varnostnem parametru $k$ in sekundarnem varnostnem parametru
$k'$, ki je deterministično izračunan iz $k$ (v praksi zadostuje kar $k' = 100$). Za velikost $L$
skupine podpisnikov $P$ predpostavimo, da je polinomska v $k$.

Shemo obravnavamo v modelu slučajnega oraklja, tako da imajo vsi podpisniki dostop do slučajnega
oraklja $H$. Na dogovorjen način ga uporabijo, da ustvarijo pet neodvisnih orakljev
\begin{align*}
    H_1, H_2: \{0, 1\}^* &\rightarrow \{0, 1\}, \\
    H_3, H_5: \{0, 1\}^* &\rightarrow \{0, 1\}^{k'}, \\
    H_4: \{0, 1\}^* &\rightarrow \{0, 1\}^{2k'}.
\end{align*}

Za lažjo obravnavo si izberemo še enega izmed podpisnikov, in ga označimo z $D$. Ta podpisnik bo
vrnil končni podpis, ne ve pa nobenih dodatnih skrivnosti. Vlogo $D$ si lahko razdelijo vsi podpisniki,
a je opis sheme malce lažji, če si izberemo enega -- v tem primeru je on prejemnik vseh sporočil
ostalih podpisnikov, svoje odgovore pa potem oddaja (angl.\ \textit{broadcast}) ostalim podpisnikom.
Če si ne izberemo podpisnika $D$, si podpisniki sporočila pošiljajo v krogu (vsak pošlje svoje sporočilo
desnemu sosedu, ko prejme sporočilo levega, spet pošlje desnemu. Vsak torej pošlje $L$ sporočil, na
koncu imajo vsi vse potrebne podatke za izračune).

Kot že omenjeno, v praksi podpisniki uporabijo dobro poznano grupo in zato ne rabijo generirati
parametrov, kar se pa tiče teoretične obravnave, pa morajo sami generirati parametre podpisne sheme.
V nadaljevanju bomo predpostavili, imajo podpisniki na voljo skupno grupo $G$ reda $q$, ki jo generira
generator $g$.

\begin{primer}[Generiranje parametrov v grupi $\Z_p^*$]
    Generiranje parametrov je seveda odvisno od izbrane grupe. V primeru $\Z_p^*$ morajo vsi podpisniki
    generirati parametre $p$, $q$ in $g$. Za prva dva uporabijo algoritem $GenPrimes(k)~\ref{alg:gen}$.
    Za izvor naključnosti uporabijo zaporedje $H_1(2^k), H_1(2^k + 1), \dots$, Za generiranje elementa
    $g$ reda $q$ uporabijo izvor naključnosti $H_2(2^k), H_2(2^k + 1), \dots$
\end{primer}

\subsubsection{Generiranje ključev}
Ko ima vsak podpisnik $P_i$ ($1 \le i \le L$) na voljo skupne parametre, si enakomerno naključno
izbere svoj zasebni ključ $s_i \in \Z_q$ in izračuna
$$ 
I_i = g^{s_i}.
$$
Do tu je postopek enak, kot pri naivni verziji. Da preprečimo napad na generiranje ključev, mora
vsak podpisnik na tem mestu izbrati naključen $u_i \in \Z_q$ in izračunati zavezo
$$
X_i = g^{u_i}.
$$
Potem si podpisniki med seboj izmenjajo vrednosti $(X_i, I_i)$. Vsak na tem mestu lahko izračuna
\begin{align*}
    e &= H_3(\text{enc}(X_1) || \text{enc}(I_1) || \text{enc}(X_2) || \text{enc}(I_2) || \dots || \text{enc}(X_L) || \text{enc}(I_L)), \\
    y_i &= e s_i + u_i \bmod q,
\end{align*}
in pošlje $y_i$ vsem ostalim podpisnikom.

Na tem mestu ima vsak podpisnik $P_i$ na voljo par $(X_j, y_j)$ vseh ostalih podpisnikov ($1 \le j
\le L$). Ta par predstavlja Schnorrov podpis in s tem dokaz znanja brez razkritja znanja zasebnega
ključa $s_j$ glede na javni ključ $I_j$. Ker je izziv $e$ izračunan iz podatkov vseh podpisnikov,
onemogoča naknadne spremembe ključev.

Vsak podpisnik $P_i$ mora preveriti veljavnost vseh podpisov $(X_j, y_j)$ vseh ostalih podpisnikov
$P_j$ ($1 \le j \le L$). To stori s standardnim izračunom za preverjanje Schnorrovega
podpisa~\eqref{eq:gen-schnorr-ver}:
$$
g^{y_j} \stackrel{?}{=} X_j \cdot I_j^{e}.
$$

Ko podpisnik $P_i$ ugotovi, da so vsi podpisi veljavni, lahko izračuna Merklovo drevo, ki ima v
listih po vrsti razporejene $I_1, I_2, \dots, I_L$ in za svojo zgoščevalno funkcijo uporablja $H_4$.
Ko je drevo izračunano, $P_i$ prebere overjevalno pot $\text{pot}_i$ svojega ključa $I_i$ in vrne
javni ključ
$$ 
\text{pk}_i = (q, g, I_i, \text{pot}_i).
$$

\subsubsection{Podpisovanje}
Naj bo $S = \{P_{id_1}, P_{id_2}, \dots, P_{id_l}\}$ podskupina velikosti $l$ skupine $P$, ki želi
večstransko podpisati sporočilo $m$. To lahko storijo s tremi krogi komunikacije:
\begin{enumerate}
    \item Vsak podpisnik $P_{id_j} (1 \le j \le l)$ si izbere naključno število $r_j \in \Z_q$
        in izračuna zavezo
        $$
        X_j = g^{r_j}.
        $$
        Zavezo $X_j$ nato pošlje podpisniku $D$.
    \item $D$ izračuna skupno zavezo
        $$
        \tilde{X} = (X_1 \cdot X_2 \cdot \cdots \cdot X_l)
        $$
        in jo pošlje vsem podpisnikom $P_{id_j} (1 \le j \le l)$.
    \item Vsak podpisnik $P_{id_j} (1 \le j \le l)$ izračuna
        \begin{align*}
            e &= H_5(\text{enc}(\tilde{X}) || m || S), \\
            y_j &= e s_j + r_j \bmod q,
        \end{align*}
        in pošlje $y_j$ podpisniku $D$.
\end{enumerate}
Na tem mestu ima $D$ vse, kar potrebuje, da dokonča podpis. Najprej izračuna
$$
\tilde{y} = (y_1 + y_2 + \cdots + y_l) \bmod q
$$
in nato vrne končni podpis
$$
\sigma = (\tilde{X}, \tilde{y}).
$$

\subsubsection{Preverjanje}
Preverjanje je podobno naivni verziji, le da je treba dodatno preveriti ustreznost Merklovih
korenov. Če torej preverjevalec želi preveriti veljavnost podpisa $\sigma = (\tilde{X}, \tilde{y})$
sporočila $m$, moramo predpostaviti, da lahko dostopa do vseh javnih ključev $\{\text{pk}_{id_1},
\text{pk}_{id_2}, \dots, \text{pk}_{id_l}\}$ vseh podpisnikov $S = \{P_{id_1}, P_{id_2}, \dots, 
P_{id_l}\}$.

Najprej mora preverjevalec preveriti, da se vseh $l$ izvodov skupnih parametrov $q$ in $g$ ujema (preveri
torej, da so vsi podpisniki uporabljali enako grupo in generator). Da dokončno potrdi veljavnost in
ustreznost javnih ključev, za vsakega podpisnika $P_{id_j} (1 \le j \le l)$ s pomočjo njegovega javnega
ključa $I_{id_j}$ in overjevalne poti $\text{pot}_{id_j}$ izračuna vrednost v korenu Merklovega drevesa
$V_{id_j}$, kot prikazano v trditvi~\ref{trd:overjevalna_pot}. Ko izračuna vse vrednosti korenov
$V_{id_j}$, mora preveriti, da so res enake.

Ko se preverjevalec prepriča o veljavnosti ključev, postopa podobno kot pri naivni verziji. Najprej
mora izračunati
$$
\tilde{I}_S = (I_{id_1} \cdot I_{id_2} \cdot \cdots \cdot I_{id_l}).
$$
Če si dosledno beleži izračune, lahko preverjevalec ta del preverjanja podpisa opravi le enkrat za
vsako podskupino podpisnikov $S$.

Potem mora preverjevalec izračunati izziv $e$ s pomočjo slučajnega oraklja
$$
e = H_5(\tilde{X} || m || S)
$$
in preveriti, če velja
\begin{equation}
\label{eq:asm-check}
g^{\tilde{y}} \stackrel{?}{=} \tilde{X} \tilde{I}_S^e.
\end{equation}

Če je torej podpis $\sigma = (\tilde{X}, \tilde{y})$ veljaven podpis sporočila $m$, lahko levo stran
enačbe~\eqref{eq:asm-check} zapišemo kot
\begin{align*}
    g^{\tilde{y}} &= g^{(y_1 + y_2 + \cdots + y_l) \bmod q} \\
                  &\stackrel{\ref{trd:exp-mod-ord}}{=} g^{y_1 + y_2 + \cdots + y_l} \\
                  &= g^{e s_1 + r_1 + e s_2 + r_2 + \cdots + e s_l + r_l} \\
                  &= g^{r_1 + r_2 + \cdots r_l} g^{e (s_1 + s_2 + \cdots s_l)},
\end{align*}
kjer smo pri prehodu iz prve v drugo vrstico uporabili lastnost~\ref{trd:exp-mod-ord}.

Desno stran enačbe~\eqref{eq:asm-check} pa lahko zapišemo kot
\begin{align*}
    \tilde{X} \tilde{I}_S^e &= (X_1 \cdot X_2 \cdot \cdots \cdot X_l)
        (I_{id_1} \cdot I_{id_2} \cdot \cdots \cdot I_{id_l})^e \\
    &=(g^{r_1} \cdot g^{r_2} \cdot \cdots \cdot g^{r_l})
        (g^{s_1} \cdot g^{s_2} \cdot \cdots \cdot g^{s_l})^e \\
    &= g^{r_1 + r_2 + \cdots r_l} g^{e (s_1 + s_2 + \cdots s_l)}.
\end{align*}
Ker se leva in desna stran ujemata, enačba~\eqref{eq:asm-check} pravilno preveri pravilnost večstranskega
Schnorrovega podpisa.

\subsection{Varnost večstranskega Schnorrovega podpisa}
\label{sec:proof_multi_schnorr}
Dokazati si želimo, da je shema, definirana v zgornjem razdelku~\ref{sec:def_multi_schnorr}, \textit{varna},
torej ustreza pogojem definicije varnosti~\ref{def:asm-varnost}, kjer ima napadalec $F$ sposobnosti,
definirane v~\ref{def:asm-napadalec}. Varnost torej pomeni, da je preprečimo eksistencialno ponarejanje,
zaradi napadalčevega nadzora nad komunikacijskim omrežjem pa si lahko privošči napad z izbranimi sporočili.
Še več, ker se ukvarjamo z večstranskimi podpisi, si napadalec lahko izbere še podskupino, ki bo 
sporočilo podpisovala. Napadalec lahko torej na katerikoli točki od katerekoli podskupine zahteva
podpis kateregakoli izbranega sporočila.

Zaradi kompleksnosti napadalca, dokaz varnosti poteka tako, da najprej dokažemo ekvivalentnost med
šibkejšim napadalcem $F'$ in prvotnim napadalcem $F$, potem pa dokažemo varnost v primeru šibkega
napadalca $F'$. Še pred tem pa moramo dokazati lemo, ki je ključna za večino dokazov varnosti podpisnih
shem v modelu slučajnega oraklja.

\subsubsection{Lema o razcepu}
Lema o razcepu (angl.\ \textit{Forking lemma}) je ključen rezultat pri dokazovanju varnosti z redukcijo,
ki omogoča pretvorbo ponarejenega podpisa v rešitev problema diskretnega logaritma. Je formalno orodje,
prek katerega pridobimo več povezanih ponarejenih podpisov, ki jih lahko združimo v rešitev problema
diskretnega logaritma. Temelji na izvedbi dveh napadov z enakimi vhodnimi podatki in parametri, a
drugačnim slučajnim orakljem.

Pred dokazom leme o razcepu moramo najprej pokazati enostavnejšo lemo iz verjetnosti.

\begin{lema}
\label{lema:verjetnost}
    Naj bosta $\mathcal{X}$ in $\mathcal{Y}$ množici. Naj bo $(X, Y)$ par slučajnih spremenljivk z
    vrednostmi iz $\mathcal{X}$ in $\mathcal{Y}$, ki ima poljubno verjetnostno porazdelitev. Naj bo
    $0 < \epsilon < 1$ in naj bo $A \subseteq \mathcal{X} \times \mathcal{Y}$ podmnožica, za katero
    velja
    $$
    \Pr((X, Y) \in A) \geq \epsilon.
    $$

    Potem obstaja podmnožica $\Omega \subset \mathcal{X}$, tako da velja:
    \begin{itemize}
        \item $\Pr(X \in \Omega) > \epsilon / 2$ (posebej to pomeni, da $\Omega \neq \emptyset$) in
        \item za vsak $a$ iz $\Omega$ velja, da $\Pr((a, Y) \in A) \geq \epsilon / 2$.
    \end{itemize}
\end{lema}

\begin{dokaz}
    Za vsak $x$ iz $\mathcal{X}$ definirajmo $p(x): \mathcal{X} \rightarrow [0, 1]$ kot
    $$
    p(x) = \Pr((x, Y) \in A \text{ }|\text{ } X = x).
    $$
    Zakon o popolni verjetnosti nam pove, da velja
    $$
    \Pr((X, Y) \in A) = \sum_{x \in \mathcal{X}} \Pr(X = x) \cdot p(x) = \text{E}[p(X)].
    $$
    Ker je po predpostavki velja $\Pr((X, Y) \in A) \geq \epsilon$, velja tudi
    $$
    \text{E}[p(X)] \geq \epsilon.
    $$

    Sedaj lahko definiramo množico $\Omega$ kot
    $$
    \Omega = \{x \in \mathcal{X} \text{ }|\text{ }p(x) \geq \epsilon / 2\} \subseteq \mathcal{X}.
    $$
    Neposredno iz definicije množice sledi drugi del leme: če je $a$ v $\Omega$, mora zanj veljati
    $$
    p(a) = \Pr((a, Y) \in A \text{ }|\text{ } X = a) \geq \epsilon / 2.
    $$

    Za prvi del leme pa najprej predpostavimo obratno, da je $\Pr(X \in \Omega) \leq \epsilon / 2$.
    Ker za $x$, ki ni v $\Omega$ velja $p(x) < \epsilon / 2$ (zunaj $\Omega$ je $p(x)$ strogo manjši
    od $\epsilon / 2$), lahko pričakovano vrednost ocenimo kot
    \begin{align*}
        \text{E}[p(X)] &= \text{E}[p(X) \text{ }|\text{ } X \in \Omega] \Pr(X \in \Omega) +
            \text{E}[p(X) \text{ }|\text{ } X \notin \Omega ] \Pr(X \notin \Omega) \\
                       &\leq 1 \cdot \Pr(x \in \Omega) + \frac{\epsilon}{2} \cdot \Pr(x \notin \Omega) \\
                       &= \Pr(x \in \Omega) + \frac{\epsilon}{2} \cdot (1 - \Pr(x \in \Omega)) \\
                       &\leq \frac{\epsilon}{2} + \frac{\epsilon}{2} \cdot \left(1 - \frac{\epsilon}{2}\right) \\
                       &= \epsilon - \frac{\epsilon^2}{4} \\
                       &< \epsilon.
    \end{align*}
    To je v nasprotju z našo predpostavko, da velja $\text{E}[p(X)] \geq \epsilon$. Tako smo pokazali,
    da mora veljati $\Pr(X \in \Omega) > \epsilon / 2$, s čimer smo dokazali prvi del leme.
\end{dokaz}

Sedaj imamo vse potrebno, da dokažemo lemo o razcepu.

\begin{izrek}[Lema o razcepu~\cite{pointcheval1996forking}]
\label{izr:forking}
    Naj bo $\mathcal{S}$ podpisna shema, ki vrača dvodelne podpise $\sigma = (\sigma_1, \sigma_2)$,
    kjer $\sigma_2$ temelji na zgostitvi vhodnega sporočila $m$ in $\sigma_1$ (npr.\ pri Schnorrovem
    podpisu je ta zgostitev izziv  $e = H(\text{enc}(X) || m)$). Naj slučajni orakelj $H$ vrača zgostitve
    dolžine $n$. Varnostni parameter naj bo $k$, tako da velja $n > k$.

    Naj bo napadalec $F$ naključnostni polinomski algoritem, ki ima na voljo vse javne podatke pri
    podpisni shemi $\mathcal{S}$ in dostop do oraklja $H$. Če lahko $F$ z nezanemarljivo verjetnostjo
    vrne par $(m, \sigma)$, kjer je $\sigma$  veljaven podpis sporočila $m$, potem lahko z nezanemarljivo
    verjetnostjo z enakimi vhodnimi podatki in uporabo drugega oraklja, $F$ vrne dva para
    $(m, \sigma)$ in $(m. \sigma')$, kjer podpisa temeljita na drugačnih zgostitvah.
\end{izrek}

\begin{dokaz}[Dokaz leme o razcepu~\ref{izr:forking}]
    Naj bo napadalec $F$ napadalec naključnostni algoritem, ki teče v polinomskem času. Z $\omega$
    označimo vir naključnih bitov, do katerega ima $F$ dostop. Denimo, da med svojim napadom napadalec
    $F$ slučajnemu oraklju $H$ pošlje največ polinomsko mnogo poizvedb v varnostnem parametru $k$.
    Označimo  poizvedbe, ki jih $F$ stori $\mathcal{Q}_1, \mathcal{Q}_2, \dots, \mathcal{Q}_Q$
    (brez škode za splošnost lahko predpostavimo, da so poizvedbe med seboj različne). Pripadajoče
    odgovore oraklja na poizvedbe označimo $\rho_1, \rho_2 , \dots, \rho_Q$.

    Predpostavimo, da lahko za naključno izbran vir bitov $\omega$ in naključno izbran orakelj $H$,
    napadalec $F$ z nezanemarljivo verjetnostjo vrne par $(m, \sigma)$, kjer je $\sigma$ veljaven podpis
    sporočila $m$. Vir naključnosti 
    Verjetnost tu jemljemo nad vsemi viri bitov $\omega$ in nad vsemi orakljevimi
    odgovori $\rho_1, \rho_2, \dots, \rho_Q$. To verjetnost označimo z $\Pr_{\omega, \rho_1,
    \rho_2 , \dots, \rho_Q}$.

    Ker so odgovori oraklja izbrani iz naključne funkcije, je verjetnost, da je bila ena izmed
    poizvedb oraklju prav tista, ki je vsebovala sporočilo $m$ in delni podpis $\sigma_1$,
    nezanemarljiva. Sicer bi moral napadalec $F$ uganiti odgovor na to poizvedbo, da bi
    lahko ustvaril drugi del podpisa $\sigma_2$ (uganiti bi torej moral naključno vrednost, kar lahko
    stori z največ zanemarljivo verjetnostjo). Označimo poizvedbo s temi podatki $\mathcal{Q}_\beta
    = (m, \sigma_1)$ ($1 < \beta < Q$). Ker je poizvedb polinomsko mnogo v varnostnem parametru $k$
    in je verjetnost uspeha nezanemarljiva, morata obstajati polinom $P(k)$ in število $\beta$, za
    katera je verjetnost uspeha
    $$
    \Pr_{\omega, \rho_1, \rho_2 , \dots, \rho_Q}(\text{$F$ vrne veljaven ponaredek in } \mathcal{Q}_\beta
    = (m, \sigma_1)) \geq \frac{1}{P(k)},
    $$
    kjer je poizvedba $\mathcal{Q}_\beta$ res temeljila na sporočilu in delnem podpisu, slučajnost
    pa izvira iz vira bitov $\omega$ in orakljevih odgovorov $\rho_1, \rho_2, \dots, \rho_Q$.

    Če označimo z $\Omega$ množico vseh naključnih virov bitov $\omega$ in z $R$ množico vseh
    zaporedij $\rho_1, \rho_2, \dots, \rho_Q$ orakljevih odgovorov, lahko uporabimo lemo~\ref{lema:verjetnost},
    kjer je $X = \Omega$, $Y = R$, $A$ je dogodek, da je $F$ uspešen in $\epsilon = 1/P(k)$.
    Lema nam zagotavlja obstoj množice $\Omega_\beta$, ki vsebuje vse vire bitov $\omega$, za katere
    velja, da je verjetnost uspeha po orakljevih odgovorih $\rho_1, \dots, \rho_Q$ navzdol omejena z
    $1/2P(k)$, oz.,\ da za vir bitov $\omega$ iz množice $\Omega_\beta$ velja
    $$
    \Pr_{\rho_1, \rho_2, \dots, \rho_Q}(\text{$F$ vrne veljaven ponaredek in } \mathcal{Q}_\beta
    = (m, \sigma_1)) \geq \frac{1}{2P(k)}.
    $$

    Sedaj lahko poleg $\beta$ fiksiramo še vir bitov $\omega$ iz množice $\Omega_\beta$. To nam da
    deterministično zaporedje poizvedb $Q_1, Q_2, \dots, Q_Q$, kjer velja $\mathcal{Q}_\beta =
    (m, \sigma_1)$, napadalec pa uspe z nezanemaljivo verjetnostjo. Na tem mestu lahko ponovno
    uporabimo lemo~\ref{lema:verjetnost}, kjer je $X$ množica odgovoro na prvih $\beta - 1$ poizvedb,
    $Y$ množica odgovorov na preostalih $Q - \beta + 1$ poizvedb, $A$ je dogodek, da je $F$ uspešen
    in $\epsilon = 1/2P(k)$. To nam zagotavlja obstoj množice $R_{\beta, \omega}$, ki vsebuje vsa
    zaporedja $(\rho_1, \dots, \rho_{\beta - 1})$, za katera velja, da je verjetnost uspeha po
    preostanku odgovorov $(\rho_\beta, \dots, \rho_Q)$ večja od $1/4P(k)$. To pomeni, da za
    zaporedje $(\rho_1, \dots, \rho_{\beta - 1})$ iz množice $R_{\beta, \omega}$, velja
    $$
    \Pr_{\rho_\beta, \dots, \rho_Q}(\text{$F$ vrne veljaven ponaredek in } \mathcal{Q}_\beta
    = (m, \sigma_1)) \geq \frac{1}{4P(k)}.
    $$

    Množici $\Omega_\beta$ in $R_{\beta, \omega}$ določata možne $\beta, \omega$ in $(\rho_1, \dots,
    \rho_{\beta - 1})$, za katere si lahko izberemo dve množici odgovorov $(\rho_\beta, \dots, \rho_Q)$
    in $(\rho_\beta', \dots, \rho_Q')$, za kateri napadalec $F$ z nezanemarljivo verjetnostjo vrne
    dva para $(m, \sigma)$ in $(m, \sigma')$, kjer sta $\sigma$ in $\sigma'$ veljavna podpisa sporočila
    $m$, ki temeljita na drugačnih zgostitvah. Tu smo uporabili pogoj $n \gg \log(k)$ oz.,\ da je
    dolžina orakljevih odgovorov $n$ veliko večja od varnostnega parametra $k$. Ta pogoj nam zagotavlja,
    da je veliko različnih oraljevih odgovorov, oz.\ da je verjetnost, da se zgodi trk,
    zanemarljiva.

    Torej, z naključno izbiro $\beta$, $\omega$, $(\rho_1, \dots, \rho_{\beta - 1}), (\rho_\beta, 
    \dots, \rho_Q)$ in $(\rho_\beta', \dots, \rho_Q')$ lahko z nezanemarljivo verjetnostjo dobimo
    dva veljavna podpisa $\sigma$ in $\sigma'$ sporočila $m$, ki temeljita na drugačnih zgostitvah.
\end{dokaz}

\subsubsection{Šibek napadalec in šibka varnost}
Nasledji korak pri dokazu varnosti je definicija \textbf{šibkega napadalca} (ali nasprotnika) (angl.\
\textit{weak adversary}) $F'$. Njegov cilj je enak izvornemu napadalcu $F$. Želi si torej ponarediti podpis
s pomočjo napada z izbranim sporočilom in podskupino. Izbere si lahko podpisnika, in od njega
zahteva podpis izbranega sporočila skupaj z izbrano podskupino $S$. Napadalca se razlikujeta v tem,
da je $F'$ bistveno šibkejši, oz.\ ima manj nadzora nad potekom podpisovanja in komunikacijo med
podpisniki.

\begin{definicija}[Šibek napadalec pri ASM]
\label{def:asm-sibek-napadalec}
    Šibek napadalec $F'$ ima v modelu večstranskega podpisa podskupine z odgovornostjo naslednje
    zmožnosti:
    \begin{itemize}
        \item Preden skupina $P$ generira svoje ključe, izbere podpisnika $P_i$, ki ga bo napadel.
            Ta podpisnik bo edini, ki bo pri izvedbi podpisne sheme pošten.
        \item Ko je podpisnik $P_i$ napaden, mora $F'$ zanj priskrbeti vse vhode za algoritme in
            vse vhodna sporočila, ki jih prejme. Prav tako lahko vidi vsa poslana sporočila
            podpisnika. Efektivno pri podpisovanju sodelujeta le $F'$ in $P_i$.
        \item Po generiranju ključev, lahko šibek napadalec $F'$ od podpisnika $P_i$ zahteva podpis
            izbranega sporočila skupaj z izbrano podskupino $S$ (izvede napad z izbranim sporočilom in
            podskupino).
    \end{itemize}
\end{definicija}

Ker mora šibek napadalec za izbranega podpisnika predložiti vse vhode in komunikacijo, lahko vidimo
delovanje šibkega napadalca $F'$ kot delovanje napadalca $F$, ki pokvari vse podpisnike, razen
podpisnika $P_i$. V obeh primerih bo podpisnik $P_i$ v podpisu sodeloval le z napadalcem, ki zanj
nadzoruje celoten potek podpisovanja. To povezavo tipov napadalcev formalno opišemo v
izreku~\ref{izr:šibka-varnost}. Pred dokazom izreka pa moramo, tako kot za napadalca $F$, tudi za
šibkega napadalca definirati, kaj pomeni varnost sheme.

\begin{definicija}[Šibka varnost pri ASM]
    Naj bo $k$ varnostni parameter (ki si ga delijo vsi podpisniki). Naj bo $c > 0$ poljubna konstanta. 
    Naj bo $F'$ šibek napadalec, ki je omejen s polinomskim časom v parametru $k$ in ustreza
    definiciji~\ref{def:asm-sibek-napadalec}. Naj bo $P_i$ podpisnik, ki ga šibek napadalec napada.
    Naj bo $p$ verjetnost, da šibek napadalec $F'$ vrne trojico $(\sigma, m, S)$, za katero velja: 
    \begin{itemize}
        \item $\sigma$ je veljaven večstranski podpis sporočila $m$ s strani skupine $S$.
        \item Podpisnik $P_i$ je v skupini $S$, vendar šibek napadalec $F'$ od njega ni zahteval
            podpisa sporočila $m$ s strani skupine $S$.
    \end{itemize}
    Shemi za večstranske podpise podskupine z odgovornostjo bomo rekli, da je \textbf{šibko varna},
    če je verjetnost $p$ zanemarljiva, oz.\ če za vsakega polinomsko omejenega šibkega napadalca $F'$
    obstaja zanemarljiva funkcija $\varepsilon$, da velja
    $$
    p < \varepsilon(k).
    $$
\end{definicija}

Sedej imamo vse potrebno, da dokažemo ekvivalentnost šibke varnosti in varnosti, in s tem
ekvivalentnost šibkega napadalca in napadalca.

\begin{izrek}
\label{izr:šibka-varnost}
    Naj bo velikost $L$ skupine podpisnikov $P$ polinomsko omejena v varnostnem parametru $k$.
    Potem je večstranski Schnorrov podpis skupine velikosti $L$ iz razdelka~\ref{sec:def_multi_schnorr}
    varen natanko tedaj, ko je šibko varen.
\end{izrek}

\begin{dokaz}
    Ker se definicija varnosti in šibke varnosti razlikuje le v uporabi napadalca $F$ ali šibkega
    napadalca $F'$, je potrebno pokazati le, da sta $F$ in $F'$ ekvivalentna (ob ustrezni velikost
    skupine podpisnikov).

    $(\Rightarrow)$: Predpostavimo, da podpis ni šibko varen. Kot smo že prej smo omenili, je delovanje
    šibkega napadalca $F'$ enako delovanju napadalca $F$, ki pokvari vse podpisnike razen enega.
    Napadalec $F$ ima torej vse zmožnosti, ki jih ima šibek napadalec $F'$. Če podpis ni šibko varen,
    lahko $F$ s simulacijo $F'$, opisano zgoraj, z nezanemarljivo verjetnostjo vrne ponarejen podpis,
    torej shema tudi ni varna.

    $(\Leftarrow)$: Predpostavimo, da podpis ni varen. Obstaja torej napadalec $F$, ki je omejen
    s polinomskim časom v varnostnem parametru $k$ in uspešno izvede napad z izbranim sporočilom in
    podskupino.

    Konstruirajmo šibkega napadalca $F'$: skladno z definicijo si pred generiranjem ključev enakomerno
    naključno izbere podpisnika $P_i$ ($1 \le i \le L$), ki ga bo napadel. Od tega trenutka naprej
    pri podpisovanju sodelujeta samo podpisnik $P_i$ in šibek napadalec $F'$, ki ima vlogo vseh
    ostalih podpisnikov.

    Ko napadalec $F$ izvede napad, mora torej šibek napadalec $F'$ simulirati delovanje vseh podpisnikov
    razen $P_i$. Vedenje šibkega napadalca $F'$ je tekom napada odvisno od dejanj napadalca $F$:
    \begin{itemize}
        \item Če se napadalec $F$ odloči pokvariti kateregakoli podpisnika, ki ni $P_i$, mu mora 
            šibek napadalec $F'$ (v vlogi podpisnika) posredovati potrebne zasebne informacije. To
            lahko stori, saj simulira delovanje podpisnika.
        \item Če napadalec $F$ pokvari podpisnika $P_i$, je šibek napadalec pri svojem napadu neuspešen.
        \item Če si napadalec $F$ izbere podpisnika (razen $P_i$) za tarčo napada z izbranim sporočilom
            in podskupino, mora šibek napadalec $F'$ simulirati odziv podpisnika. To ponovno lahko
            stori, saj simulira delovanje podpisnika.
        \item Če napadalec $F$ izbere podpisnika $P_i$ za tarčo napada z izbranim sporočilom in podskupino,
            šibek napadalec $F'$ prejme zahtevek za napad od $F$ in ga posreduje podpisniku $P_i$.
            Ta podpisnik normalno pošlje svoj odgovor napadalcu $F$.
    \end{itemize}

    Ko $F$ konča svoj napad, vrne trojico $(\sigma, m, S)$. Če je to veljaven ponaredek, jo vrne tudi
    šibek napadalec $F'$. Če ponaredek ni veljaven, če $P_i$ ni eden izmed podpisnikov v $S$, ali pa
    če je $P_i$ tekom podpisovanja pokvarjen s strani $F$, je šibek napadalec neuspešen pri svojem napadu.
    Poglejmo si verjetnost, da $F'$ uspe pri svojem napadu:
    \begin{align*}
          &\Pr(\text{$F'$ vrne veljaven ponaredek}) = \\
        = &\Pr(\text{$F$ vrne veljaven ponaredek } \cap \text{ $P_i \in S$ in ni pokvarjen}) \\
        = &\sum_{i=1}^L \Pr(\text{$F'$ izbere $P_i$}) \cdot 
            \Pr(\text{$F$ vrne veljaven ponaredek } \cap \text{ $P_i \in S$ in ni pokvarjen}) \\
        = &\sum_{i=1}^L \frac{1}{L} \cdot 
            \Pr(\text{$P_i \in S$ in ni pokvarjen | $F$ vrne veljaven ponaredek}) \\
          &\cdot \Pr(\text{$F$ vrne veljaven ponaredek}) \\
        \geq &\frac{1}{L} \cdot \Pr(\text{$F$ vrne veljaven ponaredek}),
    \end{align*}
    kjer smo pri prehodu iz tretje v četrto vrstico uporabili osnovno lastnost verjetnosti, ki pravi
    $$
    \Pr(A \cap B) = \Pr(A \text{ }|\text{ } B) \cdot \Pr(B),
    $$
    pri prehodu v zadnjo vrstico pa smo upoštevali, da velja
    \begin{equation}
    \label{eq:geq1}
    \sum_{i=1}^L \Pr(\text{$P_i \in S$ in ni pokvarjen | $F$ vrne veljaven ponaredek}) \geq 1.
    \end{equation}
    Neenačba~\eqref{eq:geq1} drži, saj mora pri uspešnem ponarejanju sodelovati vsaj en nepokvarjen
    podpisnik. Če torej $F$ vrne veljaven ponaredek, bo vsaj ena od verjetnosti v zgornji vsoti enaka $1$.

    Pri ponarejanju je torej $F$ uspešen največ $L$-krat več kot $F'$, kar pomeni, da shema ni šibko
    varna, saj je velikost skupine $L$ polinomsko omejena.
\end{dokaz}

\subsubsection{Dokaz varnosti}
Ker sta napadalec in šibek napadalec ekvivalentna, je dovolj pokazati varnost v primeru šibkega
napadalca, shema pa je potem tudi varna po izreku~\ref{izr:šibka-varnost}.

\begin{izrek}[Varnost večstranskega Schnorrovega podpisa]
    Če drži predpostavka o težavnosti diskretnega logaritma pri ASM~\ref{def:dl}, je večstranski 
    Schnorrov podpis, definiran v razdelku~\ref{sec:def_multi_schnorr}, varen.
\end{izrek}

\begin{dokaz}
Želimo torej pokazati, da če obstaja šibek napadalec, ki je omejen s polinomskim časom v varnostnem
parametru $k$, potem obstaja nek algoritem $A$, ki prav tako teče v polinomskem času, s katerim lahko
prekršimo predpostavko o diskretnem logaritmu pri ASM.

Naj bo $F'$ šibek napadalec, ki uspešno izvede napad z izbranim sporočilom in podskupino, $P_i$ pa
podpisnik, ki ga napada (in je torej edini pošten podpisnik). Javni ključ podpisnika $P_i$ označimo
$I_i$.

Konstruirali bomo algoritem $A$, ki za vhod prejme javne parametre sheme in vrednost za katero želi
izračunati diskretni logaritem, torej $q, g$ in $I$, in vrne $s \in \Z_q$, da velja $g^s = I$. Ideja
je, da $A$ simulira proces podpisovanja za šibkega napadalca $F'$. Cilj algoritma $A$ je, da iz
vrnjenega ponarejenega podpisa, pridobljenega ob uspešnem napadu, razbere diskretni logaritem, ki ga
išče. Za doseganje tega cilja mora algoritem $A$ biti sposoben simulirati delovanje vseh aspektov
procesa podpisovanja (od slučajnega oraklja do podpisnika $P_i$, ki ga na začetku izbere šibek
napadalec $F'$ za svojo tarčo).

Algoritem $A$ mora najprej doseči, da sta parametra $q$ in $g$ iz predloženega problema diskretnega
logaritma točno deljeni javni parametri pri večstranskem Schnorrovem podpisu. To doseže prek
manipulacije izvora naključnosti za generiranje parametrov, torej orakljev $H_1$ in $H_2$. Ker
algoritem $A$ simulira delovanje vseh slučajnih orakljev, lahko nastavi $H_1$ in $H_2$ tako, da
bodo rezultati algoritma za generiranje ključev točno $q$ in $g$. Za konec nastavi še javni ključ
$I_i$ podpisnika $P_i$ na $I$.

Okvirno lahko delovanje šibkega napadalca $F'$ opišemo prek dveh vrst poizvedb, ki jih opravlja pri
slučajnem oraklju:
\begin{itemize}
    \item \textit{zgoščevalna poizvedba}, kjer $F'$ pošlje slučajnemu oraklju nek niz, ta pa mu
        odgovori z njegovo zgostitvijo. Algoritem $A$ lahko simulira odgovor na tovrstne poizvedbe z
        vračanjem naključnih odgovorov. Največje števiko zgoščevalnih poizvedb, ki jih opravi $F'$,
        označimo $q_{\text{zgoščevalna}}$.
    \item \textit{podpisovalna poizvedba}, kjer šibek napadalec $F'$ podpisniku $P_i$ (ki ga je
        izbral na začetku) pošlje sporočilo $m$ in skupino $S$, ta pa mu nazaj pošlje zavezo $X_i$.
        Potem mora šibek napadalec $F'$ igrati vlogo podpisnika $D$, ki agregira informacije (tekom
        podpisovanja zbere individualne zaveze in podpise, vrača pa skupne). Podpisniku $P_i$ pošlje
        skupno zavezo $\tilde{X}$, nazaj pa dobi $y_i$. Podpisovalne poizvedbe torej potekajo v dveh
        krogih in predstavljajo izvedbo napada z izbranim sporočilom in podskupino nad podpisnikom
        $P_i$, katerega delovanje mora uspešno simulirati algoritem $A$. Največje število teh poizvedb,
        ki jih opravi $F'$, označimo $q_{\text{podpisovalna}}$.
\end{itemize}

Za lažjo konstrukcijo algoritma $A$, je smotrno šibkega napadalca $F'$ še dodatno poenostaviti. Naj $\mathcal{F}$ označuje
\textit{poenostavljenega napadalca}, ki deluje popolnoma enako kot šibek napadalec $F'$, le da pred
drugim krogom podpisovalne poizvedbe $\mathcal{F}$ najprej pošlje zgoščevalno poizvedbo slučajnemu
oraklju $H_5$ z vhodom $(\text{enc}(\tilde{X}), m, S)$. Podobno, preden $\mathcal{F}$ vrne ponarejen podpis
$(\tilde{X}, \tilde{y})$ sporočila $m_F$, najprej opravi zgoščevalno poizedbo pri slučajnemu oraklju
$H_5$ z vhodom $(\text{enc}(\tilde{X}), m_F, S)$. Tako $\mathcal{F}$ vsaki podpisovalni poizvedbi in ponaredku
dodeli zgostitev.

Dodatno predpostavimo še, da poenostavljen napadalec $\mathcal{F}$ po vsaki zgoščevalni poizvedbi
shrani odgovor, zato vsako zgoščevalno poizvedbo za določen vhod opravi le enkrat. 

Poenostavljen napadalec $\mathcal{F}$ torej opravi $q_{\text{podpisovalna}}$ podpisovalnih poizvedb
in $q_{\text{zgoščevalna}} + q_{\text{podpisovalna}} + 1$ zgoščevalnih poizvedb. Skupno število
klicev oraklja $H_5$ s strani $\mathcal{F}$ označimo $q_H$. Shranjevanje odgovorov na poizvedbe je
računsko zanemarljivo, zato $\mathcal{F}$ opravi toliko dela, kot šibek napadalec $F'$ skupaj s
stroškom poizvedb. Če torej predpostavimo učinkovitost slučajnega oraklja, teče $\mathcal{F}$ v
polinomskem času, saj tudi šibek napadalec $F'$ teče v polinomskem času.

Algoritem $A$ bo za delovanje uporabljal poenostavljenega napadalca $\mathcal{F}$, saj bo tako za
vsak potencialni ponaredek in vsak napad prejel predogled rezultata prek zgostitve, ki jo vrne $H_5$.

Algoritem $A$ mora torej znati odgovoriti na vse poizvedbe poenostavljenega napadalca $\mathcal{F}$.
Na zgoščevalne poizvedbe odgovarja naključno, s pomočjo vnaprej izbranih simuliranih odgovorov $e_1,
e_2, \dots, e_{q_H}$ oraklja $H_5$. Odgovor na podpisovalno poizvedbo za podpisnika $P_i$ na vhodu
$m$ in $S$ je malce bolj zapleten. Algoritem $A$ stori sledeče:
\begin{enumerate}
    \item Shrani konfiguracijo poenostavljenega napadalca $\mathcal{F}$.
    \item \label{rewind} Si izbere naključen odgovor $e_j$ ($1 \le j \le q_H$) oraklja $H_5$, 
        v upanju, da bo to zgostitev, ki bo identificirala to poizvedbo.
    \item Si izbere naključen $y \in \Z_q$.
    \item Izračuna 
        $$
        X_i = I_i^{-e_j}g^y \bmod p
        $$
        in pošlje $X_i$ v odgovor prvega kroga poizvedbe poenostavljenemu napadalcu $\mathcal{F}$.
    \item Po prejetju $\tilde{X}$ od $\mathcal{F}$ preveri, če $e_j$ ustreza poizvedbi na vhodu
        $(\tilde{X}, m, S)$. Če da, vrne $y$, sicer se vrne na korak~\ref{rewind}.
\end{enumerate}
Zgornje je primer splošne metode za analiziranje varnosti v kriptografiji, ki se imenuje 
\textbf{previjanje nazaj} (angl.\ \textit{rewinding}). Popularna je pri analizi dokazov brez razkritja
znanja in pri dokazih varnosti v modelu slučajnega oraklja. Metoda dovoli, da prek naključnih izbir
algoritem odgovori na poizvedbo, preden je postavljena, za ceno največ toliko previjanj, kot ima
poizvedba odgovorov.

\begin{opomba}
    V kontekstu dokaza varnosti je napadalec naključnostni algoritem, omejen s polinomskim časom.
    Algoritem $A$, ki ga konstruiramo, pa je algoritem, ki tekom svojega delovanja uporablja
    napadalca kot podprogram. To algoritmu $A$ omogoča, da pozna celotno interno stanje
    napadalca, vključno z vsemi naključnimi izbirami, ki jih je napadalec opravil. Zato lahko
    shranjuje stanja napadalca, in shranjena stanja uporabi, da ga previje nazaj na prejšnje stanje.
\end{opomba}

V tem primeru je previjanje nazaj nujno potrebno za simuliranje iskrenega podpisnika $P_i$. To je zato,
ker poenostavljen napadalec $\mathcal{F}$ igra vlogo vseh ostalih podpisnikov, torej ima odločilno
vlogo pri izbiri skupne zaveze $\tilde{X}$. Na podlagi te zaveze je izračunan skupni izziv 
$e = H_5(\tilde{X} || m || S)$, ki ga $P_i$ uporabi za izračun $y_i$. 

Ko algoritem $A$ simulira podpisni proces za poenostavljenega napadalca $\mathcal{F}$, mora torej
$A$ izračunati zavezo $X_i$ in jo poslati v prvem krogu poizvedbe (namesto podpisnika $P_i$). Problem
je, da na tej točki še ni znan izziv $e$. Za uspešno simulacijo mora algoritem $A$ uganiti, kateri
odgovor oraklja $H_5$ bo ustrezal vhodu. Ker je možnih odgovorov oraklja $q_H$ in je vsaka enako
verjetna, je verjetnost, da algoritem $A$ izbere pravilen odgovor, $1/q_H$. Pričakovano število
potrebnih previjanj je torej $q_H$.

Sedaj lahko konstruiramo algoritem $A$ na podlagi poenostavljenega napadalca $\mathcal{F}$. $A$ torej
prejme instanco problema diskretnega logaritma s podatki $p, q, g$ in $I$ (kot predstavljeno v
definiciji~\ref{def:asm_dlp}).

Naslednji korak je dokaz znanja diskretnega logaritma $g^s \equiv I \pmod q$. Ta diskretni logaritem
je točno ta, ki ga algoritem $A$ išče, torej ga na tem mestu še ne pozna. Dokaz znanja lahko zaobidemo
na podoben način, kot pri odgovarjanju na podpisovalne poizvedbe. V jedru problema je ponovno dejstvo,
da mora $A$ poslati zavezo $X_i$, preden pozna izziv $e$. Algoritem $A$ mora torej ponovno uganiti
pravilen odgovor, tokrat oraklja $H_3$. Ponovno to lahko doseže s postopkom previjanja poenostavljenega
napadalca $\mathcal{F}$. Če označimo število poizvedb, ki jih $\mathcal{F}$ pošlje oraklju $H_3$ s
$q_{H_3}$, lahko izračunamo, da je pričakovano število previjanj, ki jih opravi algoritem $A$,
enako $q_{H_3}$.

Na tem mestu je vse pripravljeno, da lahko algoritem $A$ začne simulirati postopek podpisovanja in
požene poenostavljenega napadalca $\mathcal{F}$ ter odgovarja na njegove poizvedbe kot opisano zgoraj.
Ideja dokaza od tu naprej je, da ko poenostavljen napadalec $\mathcal{F}$ vrne ponarejen podpis,
algoritem $A$ uporabi previjanje in \textbf{lemo o razcepu}~\ref{izr:forking}, da pridobi še en
ponarejen podpis. Iz teh dveh ponaredkov potem lahko izlušči odgovor na dani problem diskretnega
logaritma.

Denimo, da $\mathcal{F}$ vrne ponarejen podpis $(\tilde{X}_0, \tilde{y}_0)$ sporočila $m_0$ s strani
podskupine $S_0$. Še več, ponaredek je nastal na podlagi $j_0$-te poizvedbe oraklju $H_5$ z vhodom
$(\tilde{X}_0, m_0, S_0)$ in odgovorom $e_{j_0}$. Ker $A$ pozna celotno notranje stanje poenostavljenega
napadalca $\mathcal{F}$ in si je shranjeval njegova interna stanja, lahko simulator $A$ previje
poenostavljenega napadalca $\mathcal{F}$ nazaj na začetek podpisovnja z enakim virom naključnih bitov
$\omega$. Na njegove poizvedbe odgovarja enako kot prej, le na $j_0$-to poizvedbo odgovori z novo
naključno vrednostjo $e_{j_0}'$. Delovanje napadalca $\mathcal{F}$ pred previjanjem imenujemo prvi
pogon, delovanje po previjanju pa drugi pogon.

Ker so podpisovalne poizvedbe razdeljene na dva kroga in temeljijo na zgoščevalnih poizvedbah (za
pridobitev izziva $e$), se lahko zgodi, da je $j_0$-ta zgoščevalna poizvedba opravljena sredi ene
od podpisovalnih. Označimo z $r$ število podpisovalnih poizvedb v prvem pogonu, katerih prvi krog
poenostavljen napadalec $\mathcal{F}$ opravi pred $j_0$-to zgoščevalno poizvedbo. Ker mora legitimen
ponarejen podpis podpisovati sporočilo s strani skupine, ki tega sporočila še ni podpisala, lahko
sklepamo, da nobena od podpisovalnih poizvedb v prvem pogonu ni temeljila na $j_0$-ti zgoščevalni
poizvedbi.

Ker je pri drugem pogonu uporabljen enak vir naključnih bitov $\omega$ in so odgovori na poizvedbe do
$j_0$-te enaki, nobena od prvih $r - 1$ podpisovalnih poizvedb v drugem pogonu ni temeljila na $j_0$-ti
zgoščevalni. V drugem pogonu napadalca lahko algoritem $A$ torej vrne shranjene odgovore na prvih
$r - 1$ podpisovalnih poizvedb, kar med drugim pomeni, da za te poizvedbe ni potrebno previjanje
nazaj. To pa je lahko potrebno pri $r$-ti podpisovalni poizvedbi, saj se drugi krog lahko zgodi po
$j_0$-ti podpisovalni poizvedbi. Verjetnost previjanja je $1 - 1/q_H$, kjer je $1/q_H$ verjetnost,
da je drugi krog $r$-te podpisovalne poizvedbe izveden na podlagi enake zgoščevalne poizvedbe, kot
pri prvem pogonu.

\begin{opomba}
    V dejstvo, da nobena od prvih $r - 1$ podpisovalnih poizvedb v drugem pogonu ni temeljila na
    $j_0$-ti zgoščevalni poizvedbi, temelji na predpostavki, da ne dovoljujemo hkratnih podpisovalnih
    poizvedb, ki je standardna pri dokazih z lemo o razcepu.
\end{opomba}

Če previjanje pri $r$-ti podpisovalni poizvedbi ni potrebno, potem so pogoji pri $j_0$-ti zgoščevalni
poizvedbi popolnoma enaki, kot pri prvem pogonu (enak vir naključnih bitov napadalca in enake
zgoščevalne poizvedbe). Od tu lahko zaključimo, da bo tudi $j_0$-ta poizdveba imela enak vhod:
$(\tilde{X}_0, m_0, S_0)$, odgovor pa bo seveda drugačen. Če tudi v tem pogonu poenostavljen napadalec
$\mathcal{F}$ uspešno vrne ponaredek $(\tilde{X}_1, \tilde{y}_1)$ podpisa sporočila $m_1$ s strani
podskupine $S_1$, ki temelji na poizvebi $j_0$, potem smo lahko prepričani, da velja $m_1 = m_0,
S_1 = S_0$ in $\tilde{X}_1, = \tilde{X}_0$.

Ključno, enačbi za preverjanje podpisov potem lahko zapišemo kot
\begin{align*}
    g^{\tilde{y}_0} &\equiv \tilde{X}_0 \tilde{I}_{S_0}^{e_{j_0}} \pmod p, \\
    g^{\tilde{y}_1} &\equiv \tilde{X}_1 \tilde{I}_{S_1}^{e_{j_0}'}
        \equiv \tilde{X}_0 \tilde{I}_{S_0}^{e_{j_0}'} \pmod p.
\end{align*}
Opazimo, da lahko izrazimo $\tilde{X}_0$ na dva načina, ki ju enačimo:
$$
\tilde{X}_0 \equiv g^{\tilde{y}_0}(\tilde{I}_{S_0}^{e_{j_0}})^{-1}
    \equiv g^{\tilde{y}_1}(\tilde{I}_{S_0}^{e_{j_0}'})^{-1} \pmod p.
$$
Če premečemo dobljeno enačbo, dobimo
$$
\tilde{I}_{S_0}^{- e_{j_0}'} \tilde{I}_{S_0}^{e_{j_0}} \equiv g^{\tilde{y}_0}(g^{\tilde{y}_1})^{-1} \pmod p,
$$
kar nam omogoča izraziti $\tilde{I}_{S_0}$ kot
$$
\tilde{I}_{S_0} \equiv g^{(\tilde{y}_0 - \tilde{y}_1) (e_{j_0} - e_{j_0}')^{-1}} \pmod p.
$$
To torej pomeni, da $A$ lahko izračuna diskretni logaritem $\tilde{I}_{S_0}$ kot 
$(\tilde{y}_0 - \tilde{y}_1) (e_{j_0} - e_{j_0}')^{-1}$. Še vedno pa nismo pri koncu, saj to ni
diskretni logaritem, ki nas zanima. Če podpisnike iz $S_0$ označimo $S_0 = \{P_{i_1}, \dots, P_{i_l}\}$,
potem je 
$$
\tilde{I}_{S_0} = \prod_{j=1}^l I_{i_j},
$$
kjer je vrednost, katere diskretni logaritem želimo izračunati, $I_i$, nujno ena izmed $I_{i_j}$,
saj mora podpisnik $P_i$ biti del skupine $S_0$, če želimo, da je ponaredek veljaven. $A$ mora sedaj
pridobiti diskretne logaritme vseh ostalih $I_{i_j}$ in jih odšteti iz diskretnega logaritma
$\tilde{I}_{S_0}$. Tu uporabimo dejstvo, da mora $P_i$ med generiranjem ključev pridobiti in preveriti
dokaze znanja brez razkritja znanja o diskretnih logaritmih vseh $I_{i_j}$, razen za $i_j = i$.
Spomnimo se, da poenostavljen napadalec $\mathcal{F}$ deluje tako, da si izbere tarčo (podpisnika $P_i$)
in se nato postavi v vlogo vseh ostalih podpisnikov. To pomeni, da mora dokaze znanja generirati
prav napadalec $\mathcal{F}$ (ki torej pozna vrednosti $s_{i_j}$).

Ker so dokazi znanja brez razkritja znanja o diskretnih logaritmih samo Schnorrovi podpisi, lahko
uporabimo lemo o razcepu~\ref{izr:forking}, prek katere algoritem $A$ lahko previje napadalca
$\mathcal{F}$ in prejme diskretni logaritem $s_{i_j}$, katerega znanje dokazuje $\mathcal{F}$.
Ker vsi podpisniki za dokaz uporabijo enak izziv $H_3(\text{enc}(X_1), \text{enc}(I_1), \dots,
\text{enc}(X_L), \text{enc}(I_L))$, se mora zgoditi samo en razcep, na podlagi katerega lahko
algoritem $A$ izračuna vse diskretne logaritme $s_{i_j}$.

Vprašanje, ki ostane, je časovna zahtevnost algoritma $A$. Standardne operacije podpisovanja
so polinomske v varnostnem parametru $k$, zato so polinomske tudi za $A$. Prav tako so učinkoviti
odgovori na zgoščevalne poizvedbe. Najbolj časovno zahteven del so previjanja, ki jih je lahko $q_H$
za vsak podpis in generiranje ključev. Ker je poenostavljen napadalec $\mathcal{F}$ polinomski algoritem,
lahko opravi največ polinomsko poizvedb, torej je $q_H$ polinomsko število v $k$. Časovna zahtevnost
algoritma $A$ pa je proporcionalna $q_H$, torej tudi polinomska v $k$.

Verjetnost uspeha algoritma $A$ temelji na verjetnosti uspeha poenostavljenega napadalca $\mathcal{F}$.
Ta verjetnost je pomnožena še s faktorjem, ki je obratno sorazmeren polinomu v $q_H$. Ker smo
začeli z nezanemarljivo verjetnostjo, in jo skalirali samo s polinomskimi faktorji, je verjetnost
uspeha algoritma $A$ še vedno nezanemarljiva. $A$ je torej res polinomski algoritem, ki z
nezanemarljivo verjetnostjo vrne odgovor na problem diskretnega algoritma. Ker predpostavljamo, da
tak algoritem ne obstaja, je večstranski Schnorrov podpis varen.
\end{dokaz}
