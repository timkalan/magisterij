% LTeX: language=sl-SI
\section{Sodobni pristopi k večstranskim podpisom}
\label{sec:sodobno}
Razvoj večstranskih podpisov se je začel v osemdesetih letih prejšnjega stoletja, ko sta Nakamura
in Itakura~\cite{itakura1983multi} predstavila prvi večstranski podpis. Do leta $2001$ nihče ni
uspel dokazati varnosti večstranskega podpisa, niti ni nihče predstavil formalnega modela za
tovrstne podpise. Takrat so Micali, Ohta in Reyzin~\cite{micali2001asm} predstavili večstranske
podpise podskupine z odgovornostjo in pokazali njihovo varnost v modelu slučajnega oraklja.

Glavna pomanjkljivost njihovega modela je bila, da so podpisi zahtevali tri kroge komunikacije, kar
se je za praktično uporabo izkazalo za nedopustno. Glavna težnja razvoja večstranskih podpisov je
tako postala zmanjšanje števila potrebnih krogov komunikacije. V zadnjih letih se je pojavilo več
shem, ki so obljubljale prav to~\cite{drijvers2019security, jonas2020musig2}.

Cena zmanjšanja količine potrebne komunikacije pa je očitno bila opustitev lastnosti \textit{odgovornosti}
in \textit{prilagodljivosti}. Novejši večstranski podpisi še vedno predstavljajo dokaz, da je neka
skupina skupno podpisala sporočilo, vendar ne razkrivajo, kdo je bil član skupine. V zameno za to
pa so sposobni vrniti kratke podpise, ki so v nekaterih primerih celo popolnoma enaki navadnim
Schnorrovim podpisom~\cite{jonas2020musig2}.

V nadaljevanju poglavja si najprej ogledamo varnostne probleme večstranskih podpisov z dvema
krogoma komunikacije, nato pa predstavimo MuSig2, moderno, varno in učinkovito shemo, ki je v
uporabi danes.

\subsection{Varnost večstranskih podpisov v splošnem}
\label{sec:varnost}
Tekom let je bilo predstavljenih več shem večstranskih podpisov, ki so za podpisovanje potrebovale
le dva kroga komunikacije med podpisniki in so temeljile na Schnorrovem podpisu~\cite{bagherzandi2008bcj,
ma2010mwld, syta2016cosi, maxwell2019musig}. Večina je vsebovala tudi dokaze varnosti, katerih podlaga
je prav formalni model, ki smo ga predstavili v poglavju~\ref{sec:multischnorr}, dejanski dokazi pa
so bili mnogo kompleksnejši in daljši od predstavljenega.

Glavni cilj modernih večstranskih Schnorrovih podpisov je eliminirati prvi krog komunikacije
pri podpisovanju, torej izračuna in izmenjave zavez $X_i$. Pristopov k temu je bilo več, prav vsi
pa so se izkazali za problematične, vsaj dokler ostajamo pri Schnorrovih podpisih in modelu
slučajnega oraklja.

Uporabo katerekoli novejše sheme v praksi pa so preprečili Drijvers et al.~\cite{drijvers2019security},
ko so pokazali, da so imeli prav vsi večstranski podpisi z dvema krogoma komunikacije varnostno
luknjo. Dokazali niso samo, da so bili dokazi varnosti pomanjkljivi, temveč tudi, da predstavljene
sheme niso varne (vsaj ne z uporabo poznanih metod dokazovanja). Še več, predstavili so tudi napad
na te sheme, ki ima polinomsko časovno zahtevnost. Temeljna pomanjkljivost predstavljenih dokazov
varnosti je bila, da so zanemarili možnost, da napadalec izvede več napadov hkrati. Če je napadalec
previt, medtem ko izvaja še en napad, lahko to izkoristimo, da pridobimo še en podpis, ki temelji
na isti zavezi, kar privede do razkritja zasebnega ključa poštenega podpisnika~\cite{drijvers2019security}.

Seveda pa je dokaz nevarnosti zajel samo do tedaj poznane metode. Kasneje je raziskovalcem uspelo
ustvariti podpis, ki zaobide težave prejšnjih shem in je (zaenkrat) dokazano varen. Podpis MuSig2
je predstavljal velik preboj, saj vrača podpise, ki so popolnoma enaki Schnorrovim. Med drugim to
pomeni, da lahko njihovo veljavnost preverimo z uporabo Schnorrovee enačbe za preverjanje
podpisov~\eqref{eq:gen-schnorr-ver}.

\subsection{MuSig2}
\label{sec:musig2}
Zaradi popularnosti in enostavnosti Schnorrovih podpisov, se je pojavila težnja po razvoju večstranskega
podpisa, ki je učinkovit, varen in vrača Schnorrove podpise. Obstoj takega podpisa pomeni takojšnjo
priložnost za uporabo v vseh aplikacijah, ki temeljijo na Schnorrovih podpisih (recimo Bitcoin po
nadgradnji Taproot~\cite{wuille2020bip341}). Prvi podpis, ki je izpolnjeval vse te pogoje, je bil 
MuSig2, ki so ga leta $2020$ predstavili Jonas, Ruffing in Seurin~\cite{jonas2020musig2}. Pred pojavom
MuSig2 so obstajali drugi večstranski podpisi, ki so vračali Schnorrove podpise, vendar so zahtevali
tri kroge komunikacije, ali pa so vsaj preprečevali hkratne podpise. Eden izmed njih je tudi originalni
MuSig~\cite{maxwell2019musig}, čigar nevarnost je pokazal Drijvers~\cite{drijvers2019security}.

V tem razdelku predstavimo glavne komponente in ideje MuSig2. Podpis je dokazano varen v modelu
slučajnega oraklja, vendar bomo dokaz varnosti izpustili, saj je zelo kompleksen. Osredotočili se
bomo na razlike med predstavljenim večstranskim Schnorrovim podpisom in MuSig2.

Za začetek predstavimo glavne lastnosti MuSig2:
\begin{itemize}
    \item Omogoča hkratno podpisovanje več sporočil s strani iste skupine podpisnikov. To je zelo
        uporabno v praksi, kjer lahko hkratno podpisovanje predstavlja velik prihranek časa.
    \item Podpira \textit{agregacijo ključev} (angl.\ \textit{key aggregation}), kar pomeni, da je
        mogoče seznam javnih ključev podpisnikov združiti v en sam javni ključ, ki je lahko potem
        uporabljen za preverjanje veljavnosti podpisov.
    \item Vrnjeni podpisi so popolnoma enaki Schnorrovim podpisom. To pomeni, da preverjevalec lahko
        ne ve, ali preverja Schnorrov podpis ali MuSig2. To lastnost omogoča sposobnost agregacije
        ključev, saj lahko preverjevalec uporabi agregirani javni ključ in enačbo za preverjanje
        Schnorrovih podpisov~\eqref{eq:gen-schnorr-ver} za preverjanje veljavnosti podpisov MuSig2.
    \item Podpisovanje zahteva le dva kroga komunikacije med podpisniki. Še več, prvi krog podpisovanja
        se lahko zgodi preden je sploh znano sporočilo, kar še dodatno pohitri dejanski proces podpisa.
    \item Časovna zahtevnost podpisovanja in preverjanja je primerljiva z navadnim Schnorrovim
        podpisom.
    \item V primerjavi z opisanim večstranskim Schnorrovim podpisom, MuSig2 ne omogoča prilagodljivosti
        in odgovornosti. Torej za dano skupino podpisnikov se lahko podpišejo le vsi hkrati, njihov
        podpis priča o skupnem strinjanju, preverjrevalec ne more izvedeti, kdo je dejansko podpisoval.
        MuSig2 torej omogoča večjo zasebnost podpisnikov, kar je za nekatere aplikaicje zaželena lastnost.
\end{itemize}

\subsubsection{Skupni parametri}
Ta del podpisa je praktično enak Schnorrovemu večstranskemu podpisu. Podpisniki si izberejo varnostni
parameter $k$. Na podlagi tega parametra morajo si izberejo grupo $G$ reda $q$ z generatorjem $g$, v
kateri je problem diskretnega logaritma težak.

Poleg grupe podpisniki potrebujejo še tri zgoščevalne funkcije
$$
H_{agg}, H_{non}, H_{sig}:\{0, 1\}^* \rightarrow \Z_q,
$$
kjer s pomočjo funkcije $H_{agg}$ agregirajo javne, funkcija $H_{non}$ služi za agregiranje enkratnih
vrednosti (angl.\ \textit{nonces}), ki jih potrebujejo tekom podpisa, funkcija $H_{sig}$ pa služi
za generiranje podpisov.

Krajše lahko zapišemo
$$
((G, q, g), H_{agg}, H_{non}, H_{sig}) = \mathcal{P}(k).
$$

\subsubsection{Generiranje ključev}
Na tem mestu se pojavi prva velika prednost MuSig2. Večstranski Schnorrov podpis je preprečil napad
na generiranje ključev tako, da je od podpisnikov zahteval dokaze znanja brez razkritja znanja in
konstrukcijo Merklovega drevesa. To je zahtevalo veliko komunikacje med podpisniki, celo linearno
v njihovem številu. MuSig2 se s tem napadom sooči kasneje, med podpisovanjem. To omogoča bistveno
enostavnejšo fazo generiranja ključev, ki je povsem enaka generiranju ključev pri navadnem
Schnorrovem podpisu.

Vsak podpisnik $P_i$ ($1 \le i \le L$) si izbere naključen zasebni ključ $s_i \in \Z_q$ in izračuna
svoj javni ključ
$$
I_i = g^{s_i}.
$$
Zapišemo lahko
$$
(I_i, s_i) = \mathcal{G}(G, q, g).
$$

Da je MuSig2 dejansko lahko uporabljen kot navaden Schnorrov podpis, je potrebna še \textit{agregacija}
javnih ključev. Naj $\{I_1, I_2, \dots, I_L\}$ označuje množico vseh javnih ključev vseh podpisnikov v
skupini $P$. Za agregacijo ključev najprej vsak podpisnik $P_i$ z javnim ključem $I_i$ ($1 \le i \le L$)
izračuna \textit{koeficient agregacije} $c_i$:
$$
c_i = H_{agg}(\{\text{enc}(I_1), \text{enc}(I_2), \dots, \text{enc}(I_L)\} || \text{enc}(I_i)).
$$
\textit{Agregirani javni ključ} skupine $P$ je nato
$$
\tilde{I} = \prod_{i=1}^L I_i^{c_i}.
$$

Agregacijo lahko izvede torej kdorkoli, ki pozna vse javne ključe podpisnikov. Če to delajo podpisniki
sami, je tu potreben en krog komunikacije. Funkcijo agregacije označimo z $\mathcal{A}$, torej
lahko zapišemo
$$
\tilde{I} = \mathcal{A}(I_1, I_2, \dots, I_L) = \prod_{i=1}^L I_i^{c_i}.
$$

\subsubsection{Podpisovanje}
Ker vsa varnost MuSig2 sloni na podpisovanju, je ta del malo kompleksnejši, a vseeno zelo učinkovit.
Kot omenjeno, podpisovanje poteka v dveh krogih. Vsak krog je sestavljen iz korakov podpisovanja
in agregaicje, ki ju označimo $\mathcal{S}_i$ in $\mathcal{S}\mathcal{A}_i$ (za $i = 1, 2$). Po
komunikaciji je potreben še zadnji korak podpisovanja, ki ga označimo z $\mathcal{S}_3$.

Velika varnostna pomanjkljivost preteklih večstranskih podpisov z dvema krogoma komunikacije je bila
izbira ene same naključne vrednosti tekom podpisovanja. MuSig2 uporablja več takšnih vrednosti, pri
sami količini pa dopušča prosto izbiro. Število naključnih vrednosti bomo označili z $\nu \geq 2$.
\begin{itemize}
    \item \textbf{Prvi krog}:
        Vsak podpisnik $P_i$ ($1 \le i \le L$) si izbere $\nu$ naključnih
        vrednosti $r_{1, i, j} \in \Z_q$ ($1 \le j \le \nu$) in izračuna zaveze
        $$
        X_{1, i, j} = g^{r_{1, i, j}},
        $$
        ki jih pošlje podpisniku $D$. Ta torej od vsakega podpisnika $P_i$ prejme množico zavez
        $\{X_{1, i, j} | 1 \le j \le \nu\}$, iz katerih za vsak $j$ $(1 \le j \le \nu)$ izračuna
        $$
        X_j = \prod_{i=1}^L X_{1, i, j}.
        $$
        Končni rezultat prvega kroga so zaveze $\{X_j | 1 \le j \le \nu\}$.

        Zapišemo lahko torej
        \begin{align*}
            (X_{1, i, 1}, X_{1, i, 2}, \dots, X_{1, i, \nu}) &= \mathcal{S}_1(P_i), \\
            (X_1, X_2, \dots, X_\nu) &= \mathcal{S}\mathcal{A}_1(X_{1, i, j} | 1 \le i \le L, 1 \le j \le \nu).
        \end{align*}
        Uporabna lastnost tega koraka je, da lahko podpisniki izračunajo zaveze $X_{1, i, j}$, preden je
        sploh znano s kom bodo podpisovali in preden se izbere sporočilo $m$. Ko je enkrat skupina
        ustvarjena, lahko izvedejo še korak $\mathcal{S}\mathcal{A}_1$, ponovno brez vednosti o
        sporočilu $m$.

    \item \textbf{Drugi krog}:
        Vsak podpisnik na te mestu potrebuje svoj zasebni ključ $s_i$, agregirani javni ključ
        $\tilde{I}$, ki ga lahko izračuna z algoritmom $\mathcal{A}$ in svoj koeficient agregacije
        $c_i$. Prav tako potrebujejo rezulat prvega kroga, torej zaveze $\{X_j | 1 \le j \le \nu\}$
        in sporočilo $m$. Vsak podpisnik $P_i$ ($1 \le i \le L$) izračuna
        \begin{align*}
            b &= H_{non}(\text{enc}(\tilde{I}) || \text{enc}(X_1) || \text{enc}(X_2) || \dots || \text{enc}(X_\nu) || m), \\
            \tilde{X} &= \prod_{j=1}^\nu X_j^{b^{j-1}}, \\
            e &= H_{sig}(\text{enc}(\tilde{I}) || \text{enc}(\tilde{X}) || m), \\
            y_i &= e s_i c_i + \sum_{j=1}^\nu b^{j-1} r_{1, i, j} \bmod q.
        \end{align*}
        Krajše zapišemo
        $$
        y_i = \mathcal{S}_2(P_i, s_i, \tilde{I}, X_1, X_2, \dots, X_\nu, m).
        $$
        Vrednosti $y_i$ so poslane podpisniku $D$, ki izračuna vsoto
        $$
        \tilde{y} = \sum_{i=1}^L y_i \bmod q = \mathcal{S}\mathcal{A}_2(y_i | 1 \le i \le L).
        $$

    \item \textbf{Končni korak}:
        Če je podpisnik $D$, tisti, ki bo vrnil podpis, ima na tem mestu vse potrebne podatke.
        Enostavno vrne podpis
        $$
        \sigma = (\tilde{X}, \tilde{y}) = \mathcal{S}_3(\tilde{X}, \tilde{y}).
        $$
        sporočila $m$.
\end{itemize}

\subsubsection{Preverjanje}
Z uporabo agregacije javnih ključev, postane preverjanje podpisa MuSig2 zelo enostavno, saj je
povsem enako preverjanju navadnega Schnorrovega podpisa. Preverjevalec prejme agregiran javni ključ
$\tilde{I}_S$ skupine podpisnikov $S$, sporočilo $m$ in podpis $\sigma = (\tilde{X}, \tilde{y})$.
Izračuna izziv
$$
e = H_{sig}(\tilde{I}_S || \tilde{X} || m)
$$ in uporabi enačbo~\eqref{eq:gen-schnorr-ver}, ki jo tu zapišemo
$$
g^{\tilde{y}} \stackrel{?}{=} \tilde{X} \tilde{I}_S^e,
$$
da preveri veljavnost podpisa. Enačba res preveri veljavnost podpisa, saj lahko za veljaven podpis
zapišemo
\begin{align*}
    g^{\tilde{y}} &= g^{\sum_{i=1}^L y_i \bmod q} \\
                  &\stackrel{\ref{trd:exp-mod-ord}}{=} g^{\sum_{i=1}^L (e s_i c_i + \sum_{j=1}^\nu b^{j-1} r_{1, i, j})} \\
    &= g^{\sum_{i=1}^L e s_i c_i} \cdot g^{\sum_{i=1}^L(\sum_{j=1}^\nu b^{j-1} r_{1, i, j})} \\
    &= \prod_{i=1}^L g^{e s_i c_i} \cdot \prod_{j=1}^\nu (\prod_{i=1}^L g^{r_{1, i, j}})^{b^{j-1}} \\
    &= \prod_{i=1}^L I_i^{e c_i} \cdot \prod_{j=1}^\nu X_j^{b^{j-1}} \\
    &= \tilde{I}_S^e \cdot \tilde{X} \\
    &= \tilde{X} \tilde{I}_S^e.
\end{align*}

\subsubsection{Varnost}
Varnost pri podpisu MuSig2 ponovno pomeni nemoč napadalca, da ponaredi katerikoli podpis, pri katerem
sodeluje vsaj en nepokvarjen podpisnik. Varnost dokažemo na podoben način kot pri večstranskem Schnorrovem
podpisu, vendar dokaz presega meje magistrske naloge. Na tem mestu bomo zato predstavili le osnovne
ideje, na katerih dokaz sloni.

Podlago za varnost Schnorrovega podpisa predstavlja težavnost problema diskretnega logaritma. Pri
MuSig2 pa je za osnovo uporabljen soroden problem, ki ga imenujemo \textit{problem še enega diskretnega logaritma}
(angl.\ \textit{one-more discrete logarithm problem})~\cite{bellare2003omdl}. Zasnova je enaka kot
pri PDL. Imamo grupo z operacijo množenja, generator $g$ in element grupe $I$, zanima pa nas diskretni
logaritem $s = \log_g I$. Razlika pri problemu še enega diskretnega logaritma je, da imamo na voljo
oraklja, ki nam lahko pove $q$ rešitev problema diskretnega logaritma, mi pa želimo izračunati $q + 1$
problemov. Predpostavljamo, da ne obstaja polinomski naključnostni algoritem, ki to lahko stori z
nezanemarljivo verjetnostjo.

Pod to predpostavko lahko v modelu slučajnega oraklja dokažemo, da je MuSig2 varen, če vsak podpisnik
pri podpisovanju uporabi vsaj štiri naključne vrednosti ($\nu \geq 4$). Če dodatno predpostavimo še
\textit{model algebraičnih grup} (angl.\ \textit{algebraic group model})~\cite{fuchsbauer2018agm},
ki predpostavlja, da so napadalci algebraični, torej za vsak element grupe, ki ga izračunajo, podajo
njegov algebraični opis (na podlagi do sedaj znanih elementov), potem lahko dokažemo, da je MuSig2
varen tudi za $\nu = 2$.

Jedro dokaza je ponovno lema o razcepu~\ref{izr:forking} v kombinaciji s previjanjem. Dokaz pa je še
dodatno kompleksnejši, ker dokazuje tudi varnost v če podpisniki sodelujejo pri več podpisih naenkrat,
česar osnovna lema o razcepu ne omogoča. V tem primeru varnost omogoča dejstvo, da podpisniki izberejo
več naključnih vrednosti, ki jih združijo s pomočjo zgoščevalne funkcije $H_{non}$.
