% LTeX: language=sl-SI
\section{Pregled skupinskih podpisov}
\label{sec:skpine}
Ko pridemo do podpisovanja skupin, si lahko zamislimo več različnih rešitev. Micali~\cite{micali2001asm} 
definira dve lastnosti oz.\ spektra, ki jim lahko zadošča podpis skupine:
\begin{itemize}
    \item \textbf{Prilagodljivost} (angl.\ \textit{flexibility}): Popolnoma prilagodljiv podpis skupine
        je takšen, ki ga lahko proizvede katerakoli podskupina originalne skupine podpisnikov. Ko je podpis 
        preverjen, se mora tisti, ki ga je preveril, odločiti, če je ustrezen del skupine podal svoj podpis. 
        Popolnoma neprilagodljiv podpis bi bil takšen, ki ga lahko ustvari le celotna skupina ali pa
        katerkoli član v imenu celotne skupine.
    \item \textbf{Odgovornost} (angl.\ \textit{accountability}): Če lahko iz podpisa ugotovimo, kateri člani 
        so sodelovali pri ustvarjanju, nam podpis omogoča odgovornost. Ta lastnost je lahko zaželena, če 
        se želimo prepričati, ali je ustrezen del skupine sodeloval pri podpisu (npr.\ ali je pri podpisovanju 
        sodeloval generalni direktor podjetja). V drugih primerih pa si želimo anonimnost posameznih članov 
        (npr.\ če bi generiranje podpisa predstavljalo nekakšno tajno glasovanje, bi želeli vedeti samo, koliko 
        članov je sodelovalo).
\end{itemize}

V nadaljevanju bomo skupino potencialnih podpisnikov (torej podpisnikov, ki imajo možnost sodelovati pri 
podpisovanju) označili s $P = P_1, \dots, P_L$, kjer ima skupina $L$ članov. Dejanski podpis pa bo 
ustvaril samo del skupine $S \subseteq P$.

\subsection{Skupinski podpisi}
\textbf{Skupinski podpis} (angl.\ \textit{group signature}) v imenu celotne skupine $P$ ustvari en 
anonimen član. To torej pomeni, da je podpis popolnoma neprilagodljiv, saj ni možno prisiliti skupine,
da bi podpis ustvaril več kot en član. Prav tako v splošnem noben član, niti tisti, ki preverja podpis,
ne more ugotoviti, kdo je podpis ustvaril. Da skupinski podpisi omogočijo vsaj delno odgovornost, 
skupina določi \textit{vodjo skupine}, ki ima možnost razkriti identiteto podpisnika, če pride do 
težav. V tem primeru seveda vodja predstavlja atraktivno tarčo za napad. Skupinske podpise sta si
zamislila Chaum in van Heyst~\cite{chaum1991group}.

\begin{primer}
    Skupinski podpisi so zelo uporabni v primerih, ko moramo vedeti samo, da neka oseba pripada skupini.
    Primer je recimo dostop do varovanih območij, kjer je neprimerno, da bi natančno sledili vsem
    posameznikom, vseeno pa mora biti dostop omejen samo zaposlenim.
\end{primer}

\subsection{Pragovni podpisi}
Če želimo zagotoviti, da se s podpisom strinja zadosten delež skupine, lahko uporabimo \textbf{pragovni 
podpis} (angl.\ \textit{threshold signature}). Ta nam omogoča določeno mero prilagodljivosti, saj lahko 
katerikoli zadosten delež skupine ustvari podpis. Še vedno je nemogoče upoštevati morebitno hierarhično
strukturo skupine. Po definiciji pragovnih podpisov, ti ne omogočajo odgovornosti, nekateri celo
zagotavljajo popolno anonimnost podpisnikov.

Večina pragovnih podpisov temelji na interpolaciji polinoma $(l - 1)$-stopnje z $l$ točkami. Podpis je 
potem ustvarjen s pomočjo vrednosti polinoma v neki točki. Po interpolaciji se informacija o tem, točno 
katere točke smo uporabili, izgubi. Take podpise imenujemo tudi \textit{$l$-od-$L$ sheme}.

\begin{primer}
    Denimo, da ima banka sef, v katerem hrani vse pomembne dokumente in denar. Zaradi
    izjemne pomembnosti sefa si ne želimo, da bi ga lahko odprla katerakoli posamezna oseba. Če 
    osebje banke uporabi pragovni podpis, lahko zagotovi, da je pri odpiranju sefa vedno prisotnih
    več ljudi, vseeno katerih. Vsak zaposleni dobi točko interpolacije, ko se jih zbere dovolj,
    lahko skupaj ugotovijo vrednost polinoma v vnaprej določeni točki in odklenejo sef.
\end{primer}

\subsection{Večstranski podpisi}
\label{sec:multisig}
Za nekatere uporabe podpisov, bi si od njih želeli podobne lastnosti, kot jih ima večstranski ročen podpis. 
Pri njem lahko hitro preberemo podpisnike, torej imamo popolno prilagodljivost. Vidimo lahko seznam 
podpisnikov, torej lahko presodimo, če so med njimi tisti, ki smo jih želeli. Prav tako podpisniki nosijo 
popolno odgovornost, saj na papirju piše njihovo ime. 

Podoben učinek bi z digitalnimi podpisi lahko dosegli, če bi namesto enega podpisa skupine, od članov 
zbrali individualne podpise in jih nanizali v seznam. Dobili bi torej digitalni podpis skupine, ki 
ponuja popolno prilagodljivost in odgovornost. Težava je samo, da je dolžina podpisa (in s tem čas 
preverjanja) proporcionalna številu podpisnikov. \textbf{Večstranski podpisi} (angl.\textit{multisignatures})
ohranijo lastnosti seznama podpisov, rezultat sheme je pa en sam podpis, ki je enako dolg ne glede 
na število podpisnikov, prav tako je od števila neodvisen čas preverjanja. Tega s seznamom podpisov 
ni mogoče doseči, saj tako dolžina podpisa, kot čas preverjanja podpisa rasteta linearno s številom 
podpisnikov (vsak doda en podpis seznamu, ki ga je potrebno preveriti). Večstranski podpisi so torej 
odlična posplošitev ročnih podpisov skupin, ki vseeno ohrani učinkovito preverjanje.

Še ena pomembna lastnost večstranskih podpisov je, da nekatere sheme vračajo podpise, ki so kar se
preverjanja tiče, popolnoma zamenljivi z običajnimi digitalnimi podpisi. To omogoča enostavno
nadgradnjo obstoječih sistemov, ki že uporabljajo digitalne podpise.

\begin{primer}
    Recimo, da imamo nek organ, ki izdaja certifikate avtentičnosti uporabnikov (npr.\ potrjuje 
    avtentičnost javnih ključev). Za večjo robustnost in varnost, je lahko ta organ razporejen 
    na več strežnikov. Tako preprečimo razpad sistema v primeru izpada enega strežnika. Zato je 
    torej tudi pomembno, da certifikacijo uporabnika potrdi nekaj strežnikov, ne pa nujno vsi. 
    Tu lahko torej neka podskupina vseh strežnikov organa skupaj izda en večstranski podpis, ki 
    potrjuje avtentičnost uporabnika.
\end{primer}

\subsection{Agregirani podpisi}
Agregirani podpisi imajo zelo podobne lastnosti kot večstranski. Poleg standardnih algoritmov pri digitalnih
podpisih za ustvarjanje parametrov $\mathcal{P}$, ustvarjanje ključev $\mathcal{G}$, podpisovanje 
$\mathcal{S}$ in preverjanje $\mathcal{V}$, imajo agregirani podpisi (angl.\ \textit{aggregate signatures})
še dodaten algoritem za združevanje podpisov $\mathcal{C}$. Ta algoritem prejme seznam trojic javnih
ključev, sporočil in podpisov, vrne pa en sam podpis. Od večstranskih podpisov se razlikuje po tem,
da sporočila niso nujno vsa enaka~\cite{boneh2011aggregate}.

\begin{primer}
    Recimo, da avtentičnost ključev posameznih podpisnikov preverja niz centrov za certificiranje
    podpisov, kjer vsak naslednji center jamči za avtentičnost centra pod njem. Prvi center zajamči
    avtentičnost našega ključa, naslednji zajamči avtentičnost centra in tako naprej. Ko bi želel
    nekdo prejeti naš certificiran javni ključ, bi moral preveriti veljavnost podpisov vseh centrov,
    s pomočjo agregiranih podpisov bi to lahko storil z enim samim preverjanjem.
\end{primer}

\section{Večstranski Schnorrov podpis}
\label{sec:multischnorr}
Micali et al.~\cite{micali2001asm} so prvi definirali formalni model za večstranske podpise in podali
formalni dokaz varnosti za podpisno shemo v tem modelu. Zamislili so si formalni model za večstranske
podpise in ga poimenovali \textbf{večstranski podpis podskupine z odgovornostjo} (angl.\
\textit{Accountable-Subgroup Multisignature (ASM)}). V tem razdelku predstavimo njihov model, primer
večstranskega podpisa, ki temelji na Schnorrovem podpisu, in argumentiramo varnost podpisa oz.\
modela na sploh.
