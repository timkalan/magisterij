% LTeX: language=sl-SI
\section{Schnorrov podpis}
\label{sec:schnorr}
Eden izmed najenostavnejših, dokazano varnih podpisov je \textit{Schnorrov podpis}~\cite{schnorr1989sig}.
Kot vsi podpisi, tudi ta potrebuje štiri algoritme: za ustvarjanje javnih parametrov, ustvarjanje ključa, 
podpisovanje sporočil in preverjanje podpisa.

Čeprav je Schnorr originalno~\cite{schnorr1989sig} opisal podpis v multiplikativnih grupah naravnih
števil modulo $p$, v shemi ni nič, kar bi preprečilo delovanje v poljubnih grupah. Zares
je Schnorrov podpis mogoče posplošiti na katerekoli končne grupe, kjer obstaja učinkovit algoritem
za množenje in je problem diskretnega logaritma~\ref{def:dl} težek.
\begin{itemize}
    \item \textbf{Parametri}:
    Naj bo $G$ končna grupa reda $p$. V njej si izberemo element $g$ reda $q$, pri čemer mora
    biti $q$ dovolj veliko praštevilo (njegova velikost je odvisna od varnostnega parametra $k$).
    V splošnem pa sta lahko $p$ in $q$ tudi enaka. Ker računanje potega v podgrupi, ki jo določa $g$,
    se običajno privzame, da je $G$ kar grupa reda $q$, ki jo generira $g$.

    Poleg grupe $G$ si morata podpisnik in preverjevalec izbrati še varno kriptografsko zgoščevalno
    funkcijo $H : \{0, 1\}^* \rightarrow \Z_q$. Funkcijo v dokazu varnosti modeliramo kot naključno,
    v praksi mora zadoščati vsaj lastnostim iz definicije~\ref{def:hash}. Velikost kodomene te funkcije
    definira velikost končnega podpisa. Iz zgostitve, dolge $\log_2 q$ bitov, dobimo podpis, dolg
    $2 \log_2 q$ bitov~\cite{stinson2023crypto}.
    \item \textbf{Ključ}:
    Za ustvarjanje ključa si izberemo naključno število $s \in \Z_q$ in izračunamo
    $$
    I = g^s
    $$
    z uporabo učinkovitega algoritma za množenje. Javni ključ $I$ je torej element grupe $G$, zasebni ključ
    $s$ pa je element $\Z_q$. Ker smo predpostavili, da je v grupi $G$ problem diskretnega logaritma
    težek, iz javnega ključa $I$ ni mogoče pridobiti zasebnega ključa $s$.

    \item \textbf{Podpis}:
    Za podpis sporočila $m \in \{0, 1\}^*$ si najprej izberemo naključno število $r \in \Z_q$ in izračunamo
    \textit{zavezo} (angl.\ \textit{commitment})
    $$
    X = g^r.
    $$
    Ta korak je popolnoma enak kot pri ustvarjanju ključa, vendar ima pomembno razliko. Zasebni
    ključ se ne spreminja, pri izbiri $r$ pa je potrebno paziti, da je ta res naključna, in da se
    $r$ ne ponovi (glej opombo~\ref{opomba:nonce}).

    Potem z uporabo zgoščevalne funkcije $H : \{0, 1\}^* \rightarrow \Z_q$ izračunamo \textit{izziv}
    (angl.\ \textit{challenge})
    $$
    e = H(\text{enc}(X) || m).
    $$
    Za konec je potrebno izračunati še
    $$ 
    y = es + r \bmod q,
    $$
    podpis sporočila $m$ pa je potem par $(X, y)$ oz.\
    $$ 
    \mathcal{S}(s, m) = (X, y).
    $$

    \item \textbf{Preverjanje}:
    Za preverjanje veljavnosti podpisa $(X', y')$ sporočila $m$ je potrebno najprej izračunati
    $$
    e' = H(\text{enc}(X') || m)
    $$
    in nato preveriti, če velja
    $$
    g^{y'} \stackrel{?}{=} X' \cdot I^{e'}.
    $$
\end{itemize}

Ker Schnorrov podpis deluje v skoraj poljubnih končnih grupah, je zelo prilagodljiv in uporaben.
V zadnjem času je precej popularna uporaba Schnorrovih podpisov v \textit{eliptičnih grupah}.
Te omogočajo izbiro manjših parametrov, kar naredi podpis bolj časovno in prostorsko učinkovit.

\begin{opomba}
\label{opomba:nonce}
    V primeru, da je enak $r$ uporabljen večkrat, podpisnik tvega, da lahko napadalec iz dveh njegovih
    podpisov izračuna njegov zasebni ključ $s$. Naj bosta $(X_1, y_1)$ in $(X_2, y_2)$ podpisa sporočil
    $m_1$ in $m_2$. Potem velja
    $$
    y_1 - y_2 = e_1 s + r_1 - e_2 s - r_2 = (e_1 - e_2)s + (r_1 - r_2).
    $$
    V primeru, da sta $r_1$ in $r_2$ enaka, lahko napadalec z enostavnim izračunom inverza $(e_1 - e_2)$
    pridobi zasebni ključ $s$ (za izračun $e_1$ in $e_2$ ima napadalec dovolj informacij, saj je
    izbrana zgoščevalna funkcija javna).

    Izkaže se, da je lahko problematična že uporaba generatorja naključnih števil za pridobivanje
    $r$, ki ne vrača enakomerno porazdeljenih števil. Če napadalec dobi dovolj veliko količino
    sporočil in podpisov, lahko v tem primeru reši \textit{problem skritega števila} in pridobi
    zasebni ključ~\cite{tibouchi2017attacks}.
\end{opomba}

\subsection{Varnost Schnorrovega podpisa}
\label{sec:schnorr-sec}
Ko govorimo o varnosti Schnorrovega podpisa, imamo v mislih odpornost sheme proti eksistencialnem
ponarejanju, kjer napadalec lahko za katerokoli sporočilo dobi veljaven podpis. Želimo torej, da
napadalcu ne uspe ponarediti podpisa za nobeno sporočilo, ki še ni bilo podpisano.

Kot omenjeno na začetku poglavja, je varnost Schnorrovega podpisa odvisna od težavnosti problema
diskretnega logaritma. Varnost Schnorrovega podpisa je zato odvisna od varnosti grupe, v kateri deluje.
Cilj tega razdelka je pokazati, da je Schnorrov podpis varen, če je problem diskretnega logaritma
težek.

\begin{definicija}[Varnost podpisne sheme]
    Naj bo $k$ varnostni parameter. Naj bo $F$ napadalec, ki deluje v polinomskem času. Definirajmo
    \textit{podpisovalni eksperiment} $\text{Sign}_F(k)$:
    \begin{enumerate}
        \item Na podlagi varnostnega parametra se definira par ključev $(I, s)$, kjer je $I$
            javni ključ, $s$ pa zasebni ključ.
        \item Napadalec $F$ prejme javni ključ $I$. Od nekega podpisnika $P$ lahko zahteva podpise
            poljubno izbranih sporočil.
        \item Napadalec $F$ podpisniku $P$ lahko pošlje sporočilo $m$, v odgovor pa dobi podpis $\sigma$.
            To lahko stori poljubno mnogokrat.
        \item Napadalec $F$ si izbere sporočilo $m'$ in zanj izračuna podpis $\sigma'$. Eksperiment
            vrne $1$, če je podpis $\sigma'$ veljaven postup za sporočilo $m'$, sicer vrne $0$.
    \end{enumerate}
    Cilj napadalca je torej ponarediti podpis za sporočilo, ki še ni bilo podpisano, brez da bi poznal
    zasebni ključ podpisnika. Podpisna shema je varna , če je verjetnost, da izvedba eksperimenta
    $\text{Sign}_F(k)$ vrne $1$, zanemarljiva v varnostnem parametru $k$.
\end{definicija}

Varnost Schnorrovega podpisa je mogoče dokazati tako, da pokažemo varnost Schnorrove identifikacijske
sheme, opazimo, da nam uporaba Fiat-Shamirjeve hevristike vrne ravno Schnorrov podpis, in pokažemo, da
je ta transformacija varna. Najprej pa moramo seveda definirati, kaj pomeni, da je identifikacijska
shema varna in kako povežemo identifikacijsko shemo s sporočilom. V osnovi bomo rekli, da je shema
varna, če napadalec ne more prepričati preverjevalca, da je on pravi dokazovalec, brez da bi poznal
zasebni ključ. To mora veljati tudi v primeru, da napadalec lahko ">posluša"< (torej vidi sporočila)
več pogovorov med dokazovalcem in preverjevalcem.

\begin{definicija}[Varnost identifikacijske sheme]
\label{def:id-sec}
    Naj bo $k$ varnostni parameter in $T_{sk}$ orakelj, ki ne prejme nobenega vhodnega podatka, ob klicu
    pa vrne \textit{transkript} ene izvedbe identifikacijske sheme. Transkript predstavlja eno izvedbo
    identifikacijske sheme z izbranimi javnimi parametri. Formalizira napadalčevo sposobnost
    ">prisluškovanja"< in mu omogoča, da pridobi vsa izmenjana sporočila med dokazovalcem in
    preverjevalcem. Naj bo $F$ napadalec, omejen s polinomskim časom. Definirajmo 
    \textit{identifikacijski eksperiment} $\text{Id}_F(k)$:
    \begin{enumerate}
        \item Na podlagi varnostnega parametra se definira par ključev $(I, s)$, kjer je $I$
            javni ključ, $s$ pa zasebni ključ.
        \item Napadalec $F$ prejme javni ključ $I$ in neomejen dostop do oraklja $T_{sk}$.
        \item Napadalec $F$ na poljubni točki pošlje zavezo $X$, v odgovor pa dobi izziv $e$. Tudi
            na tej točki lahko napadalec kliče oraklja $T_{sk}$.
        \item Napadalec $F$ izračuna odgovor $y$. Eksperiment vrne $1$, če odgovor $y$ prepriča
            preverjevalca, da komunicira s pravim dokazovalcem (torej lastnikom zasebnega ključa),
            sicer vrne $0$. V primeru Schnorrove identifikacijske sheme to ustreza preverbi, če
            velja $g^y = X \cdot I^e$.
    \end{enumerate}
    Cilj napadalca je torej pridobiti diskretni logaritem javnega ključa $I$, s čimer se lahko izdaja
    za dokazovalca. Identifikacijska shema je varna pred pasivnim napadom (ali samo varna), če je
    verjetnost, da izvedba eksperimenta $\text{Id}_F(k)$ vrne $1$, zanemarljiva v varnostnem
    parametru $k$.
\end{definicija}

\begin{izrek}[Varnost Schnorrove identifikacijske sheme~\cite{katz2014introduction}]
\label{izrek:schnorr-id-sec}
    Naj bo $G$ ciklična grupa, v kateri je problem diskretnega logaritma težek. Potem je Schnorrova
    identifikacijska shema v grupi $G$ varna.
\end{izrek}

\begin{dokaz}
    Dokaz poteka z redukcijo uspešnega napada na rešitev problema diskretnega logaritma. Naj bo napadalec
    $F$ naključnostni algoritem, ki teče v polinomskem času. Cilj napadalca je, da iz javnega ključa
    dobi zasebnega (izračuna diskretni logaritem javnega ključa). Napad poteka podobno, kot opisano v
    identifikacijskem eksperimentu iz definicije~\ref{def:id-sec}.

    Na podlagi napadalca $F$ konstruirajmo algoritem $A$, ki iz uspešnega napada napadalca $F$ na
    Schnorrovo identifikacijsko shemo pridobi rešitev problema diskretnega logaritma. Algoritem $A$
    prejme vse javne parametre sheme $(G, g, q, I)$ kot vhod. $I$ je javni ključ, za katerega želi
    napadalec izračunati diskretni logaritem. Algoritem deluje v naslednjih korakih:
    \begin{enumerate}
        \item Algoritem $A$ zažene napadalca $F$. Ta med svojim delovanjem kliče oraklja $T_{sk}$,
            ki napadalcu vrača transkripte izvedb identifikacijske sheme. Algoritem $A$ mora te
            odgovore simulirati sam, to pa lahko stori tako, da izvede identifikacijsko shemo v
            obratnem vrstnem redu:
            \begin{itemize}
                \item Najprej si izbere naključen odgovor $y \in \Z_q$ in izziv $e \in \Z_q$.
                \item Preuredi enačbo za preverjanje
                    $$
                    g^y = X \cdot I^e
                    $$
                    v enačbo za izračun zaveze
                    $$
                    X = g^y \cdot I^{-e}.
                    $$
                    Tako lahko vrne veljaven transkript $(y, e, X)$ napadalcu $F$.
            \end{itemize}
            Ker račuana v obratnem vrstnem redu, se izogne izračunu diskretnega logaritma, opraviti
            mora le eno eksponenciranje.
        \item Ko napadalec pošlje zavezo $X$, algoritem $A$ izbere enakomerno naključno število $e_1
            \in \Z_q$, jo pošlje napadalcu $F$, ta pa odgovori z odgovorom $y_1$.
        \item Algoritem $A$ ponovno zažene napadalca $F$ z enakimi parametri (torej tudi enakim
            virom naključnih bitov $\omega$) a z različnim izzivom $e_2 \in \Z_q$. Napadalec
            $F$ v tem primeru vrne odgovor $y_2$.
        \item Če sta oba odgovora veljavna, torej velja
            \begin{align*}
                g^{y_1} &= X \cdot I^{e_1} \text{ in} \\
                g^{y_2} &= X \cdot I^{e_2},
            \end{align*}
            potem lahko algoritem $A$ izračuna rešitev problema diskretnega logaritma kot
            \begin{align*}
                X = g^{y_1} \cdot I^{-e_1} &= g^{y_2} \cdot I^{-e_2} \\
                g^{y_1} \cdot g^{-y_2} &= I^{e_1} \cdot I^{-e_2} \\
                g^{y_1 - y_2} &= (g^s)^{e_1 - e_2} \\
                s &= (y_1 - y_2)(e_1 - e_2)^{-1}
            \end{align*}
    \end{enumerate}

    Da dokažemo varnost, moramo sedaj obravnavati verjetnost uspeha napadalca $F$ in algoritma $A$.
    Naj bo $V(\omega, e)$ indikator, da napadalec $F$ vrne pravilen odgovor pri napadu, če je $e$ izziv
    in $\omega$ naključni vir bitov. Označimo z $\delta_{\omega}$ verjetnost uspeha po izzivu $e$
    pri fiksni vrednosti $\omega$
    $$
    \delta_{\omega} = \Pr_e(V(\omega, e) = 1).
    $$
    S to vrednostjo lahko izrazimo verjetnost uspeha identifikacijskega eksperimenta
    $$
    \Pr(\text{Id}_F(k) = 1) = \Pr_{\omega, e}(V(\omega, e) = 1) =
        \sum_{\omega} \Pr(\omega) \cdot \delta_{\omega}.
    $$

    Spomnimo se, da algoritem $A$ uspešno izračuna rešitev problema diskretnega logaritma, če napadalec
    $F$ uspe dvakrat pri različnih izzivih $e_1$ in $e_2$. Potem lahko izrazimo
    \begin{align*}
        \Pr(A \text{ uspe}) &=
            \Pr_{\omega, e_1, e_2}(V(\omega, e_1) = 1 \land V(\omega, e_2) = 1 \land e_1 \neq e_2) \\
                                  &\geq \Pr_{\omega, e_1, e_2}(V(\omega, e_1) = 1 \land V(\omega, e_2) = 1) - \Pr(e_1 = e_2) \\
                                  &= \sum_{\omega} \Pr(\omega) \cdot \delta_{\omega}^2 - \frac{1}{q} \\
                                  &\geq (\sum_{\omega} \Pr(\omega) \cdot \delta_{\omega})^2 - \frac{1}{q} \\
                                  &= \Pr(\text{Id}_F(k) = 1)^2 - \frac{1}{q},
    \end{align*}
    kjer smo pri prehodu na predzadnjo vrstico uporabili Jensenovo neenakost~\cite{jensen}.
    Ker pa je uspeh algoritma $A$ ekvivalenten rešitvi problema diskretnega logaritma v polinomskem
    času, problem pa po predpostavki obravnavamo v grupi, kjer je problem diskretnega logaritma težek,
    lahko zaključimo, da je verjetnost uspeha algoritma $A$ zanemarljiva. Ker je velikost grupe $q$
    odvisna od varnostnega parametra $k$, je tudi člen $1/q$ zanemarljiv. Iz tega lahko zaključimo
tudi, da je verjetnost uspeha napadalca $F$ $\Pr(\text{Id}_F(k) = 1)$ zanemarljiva, kar pomeni,
    da je Schnorrov identifikacijski protokol varen.
\end{dokaz}

Kot smo videli v primeru~\ref{primer:fiat-shamir}, lahko Fiat-Shamirjevo hevristiko uporabimo, da
iz Schnorrove identifikacijske sheme pridobimo neinteraktivno verzijo. Če torej lahko pokažemo,
da je ta transformacija varna, smo dokazali varnost Schnorrovega podpisa.

\begin{izrek}[Varnost Fiat-Shamirjeve hevristike~\cite{katz2014introduction}]
\label{izrek:fiat-shamir-sec}
    Naj bo $\mathcal{S'}$ Schnorrova identifikacijska shema in $\mathcal{S}$ Schnorrova podpisna
    shema, pridobljena iz $\mathcal{S'}$ z uporabo Fiat-Shamirjeve hevristike. Če je identifikacijska
    shema $\mathcal{S'}$ varna, potem je tudi podpisna shema $\mathcal{S}$ varna, če varnost
    obravnavamo v modelu slučajnega oraklja.
\end{izrek}

\begin{dokaz}
    Naj bo napadalec $F$ na podpisno shemo $\mathcal{S}$ naključnostni algoritem, ki teče v
    polinomskem času. Naj bo število napadalčevih klicev slučajnega oraklja polinomsko omejeno
    v varnostnem parametru $k$. Zgornjo mejo označimo s $q$.

    Brez škode za splošnost predpostavimo, da bo napadalec $F$ vsak klic slučajnega oraklja na
    nekem vhodu opravil največ enkrat. Predpostavimo tudi, da če napadalec $F$ uspešno ponaredi
    podpis $(X, y)$ sporočila $m$, potem je gotovo poklical oraklja $H$ s podatki $\text{enc}(X) || m$.

    Na podlagi napadalca $F$ na podpisno shemo $\mathcal{S}$ želimo konstruirati napadalca $F'$
    na identifikacijsko shemo $\mathcal{S'}$, ki ima dostop do transkripcijskega oraklja $T_{sk}$
    in javnega ključa $I$. Napadalec $F'$ deluje v naslednjih korakih:
    \begin{enumerate}
        \item Izbere si enakomerno naključno število $j \in \{1, 2, \dots, q\}$.
        \item Požene napadalca $F$ in zanj simulira delovanje podpisne sheme $\mathcal{S}$.
            Ko napadalec $F$ opravi $i$-ti klic oraklja $H$ z vhodom $\text{enc}(X_i) || m_i$,
            napadalec $F'$ odgovori na podlagi vrednosti $i$ in $j$:
            \begin{itemize}
                \item Če je $i = j$, napadalec $F'$ uporabi $X_i$ kot zavezo v identifikacijski
                    shemi, v odgovor dobi izziv $e_i$ in ga pošlje napadalcu $F$ kot odgovor na
                    klic oraklja $H$.
                \item V nasprotnem primeru is izbere enakomerno naključno število $e_i \in \Z_q$ in ga
                    pošlje napadalcu $F$ kot odgovor na klic oraklja $H$.
            \end{itemize}
            Ko napadalec $F$ zahteva podpis sporočila $m$, napadalec $F'$ pokliče transkripcijskega
            oraklja $T_{sk}$, ki mu vrne transkript $(X, e, y)$. Par $(X, y)$ vrne kot podpis
            napadalcu $F$.
        \label{step:simulate}
        \item Če napadalec $F$ uspe ponarediti podpis $(X, y)$ sporočila $m$, napadalec $F'$ preveri,
            ali velja $(X, m) = (X_j, m_j)$. Če velja, napadalec $F'$ vrne $y$ kot odgovor.
    \end{enumerate}
    V koraku~\ref{step:simulate} je simulacija podpisne sheme uspešna, saj je izziv $e$ v vsakem
    primeru izbran enakomerno naključno (le iz drugega vira), prav tako je par $(X, y)$ iz
    transkripcijskega oraklja izračunan na enak način kot veljaven podpis. Napadalec $F'$ torej ne
    loči med simulacijo in pravim podpisom. Problem nastane le, če je napadalec $F'$ že definiral
    orakljev odgovor s podatkoma $X$ in $m$, ko dobi transkript $(X, e', y)$ od transkripcijskega
    oraklja $T_{sk}$. Če v tem primeru $e$ ni enak $e'$, simulacija ne uspe. Verjetnost, da se to
    zgodi, je zanemarljiva, saj je $e$ izbran enakomerno naključno iz $\Z_q$ in je $q$ polinomsko
    omejen v varnostnem parametru $k$.

    Potrdimo še, da je rezultat napada na identifikacijsko shemo smiseln. Če v zadnjem koraku velja
    $(X, m) = (X_j, m_j)$ in napadalec $F$ uspe ponarediti podpis $(X, y)$, potem napadalec $F'$
    res lahko pošlje zavezo $X_j$, v odgovor pa dobi izziv $e_j$, na podlagi katerega napadalec na
    podpis $F$ izračuna odgovor $y$. Ta odgovor je zaradi ujemanja izziva tudi veljaven odgovor
    v zadnjem koraku identifikacijske sheme, kar pomeni, da je napadalec $F'$ uspešno napadel.

    Zadnji korak je, da pogledamo verjetnost uspeha napadalca $F'$. Naj $\delta_k$ označuje verjetnost,
    da pride do neujemanja pri odgovorih oraklja (zgoraj smo videli, da je zanemarljiva). Ker je $j$
    izbran enakomerno naključno, je verjetnost, da velja $i=j$, enaka $1/q$. Verjetnost uspešnega
    napada lahko izrazimo kot
    $$
    \Pr(F' \text{ uspe}) = \frac{1}{q} \cdot (\Pr(F \text{ uspe}) - \delta_k),
    $$
    oz.
    $$
    \Pr(F \text{ uspe}) = q \cdot \Pr(F' \text{ uspe}) + \delta_k.
    $$
    Ker je po predpostavki identifikacijska shema varna, je verjetnost uspeha napadalca $F'$
    zanemarljiva. Prav tako je tudi verjetnost $\delta_k$ zanemarljiva, $q$ pa je polinomska zgornja
    meja, kar pomeni, da je tudi verjetnost uspeha napadalca $F$ zanemarljiva. To pomeni, da je
    Schnorrov podpis varen, če je identifikacijska shema varna.
\end{dokaz}

\begin{izrek}[Varnost Schnorrovega podpisa]
    Naj bo $G$ ciklična grupa, v kateri je problem diskretnega logaritma težek. Naj bo $H$ slučajni
    orakelj, ki je dostopen vsem deležnikom. Potem je Schnorrov podpis v grupi $G$ varen.
\end{izrek}

\begin{dokaz}
    Po izreku~\ref{izrek:schnorr-id-sec} je Schnorrov identifikacijski protokol v grupi $G$ varen.
    Prav tako je po izreku~\ref{izrek:fiat-shamir-sec} varna tudi pretvorba iz Schnorrove identifikacijske
    sheme v Schnorrov podpis, če upoštevamo model slučajnega oraklja. Zaključimo torej, da je
    Schnorrov podpis varen.
\end{dokaz}

\subsection{Primer: Schnorrov podpis v \texorpdfstring{$\Z_p^*$}{Zp∗}}
Za ilustrativni primer si poglejmo, kako je bil Schnorrov podpis prvotno opisan~\cite{schnorr1989sig}.
Predstavljeni podpis je poseben primer podpisa, opisanega zgoraj, kjer je grupa $G$ multiplikativna
grupa naravnih števil modulo praštevila $p$. Za namene tega dela si ga je posebej koristno pogledati,
saj bo enaka grupa uporabljena tudi pri njegovi večstranski različici v poglavju~\ref{sec:multischnorr}.
\begin{itemize}
    \item \textbf{Parametri}:
    Najprej je potrebno generirati par praštevil $p$ in $q$, tako da $q$ deli $p - 1$. Praštevilo $p$
    definira grupo $\Z_p^*$, ki je multiplikativna grupa celih števil modulo $p$. V tej grupi je potem
    potrebno izbrati element $g$, ki je reda $q$. Za varnost je potrebno, da ima $p$ vsaj $2048$ bitov,
    $q$ pa vsaj $224$ bitov.

    Kot v splošni verziji Schnorrovega podpisa, si morata podpisnik in preverjevalec izbrati še varno
    kriptografsko zgoščevalno funkcijo $H : \{0, 1\}^* \rightarrow \Z_q$.

    \item \textbf{Ključ}:
    Za ustvarjanje para ključev je potreben izbor naključnega števila $s \in \Z_q$
    in izračun 
    $$ 
    I = g^s \bmod p.
    $$
    Ta izračun nam da par ključev
    \begin{align*}
    \text{pk} &= (p, q, g, I), \\
    \text{sk} &= s.
    \end{align*}

    \item \textbf{Podpis}:
    Za podpis enega sporočila mora podpisnik generirati naključno število $r \in \Z_q$ in izračunati 
    \textit{zavezo} 
    $$ 
    X = g^r \bmod p.
    $$ 
    Potem z uporabo funkcije $H$ izračunamo \textit{izziv} 
    $$
    e = H(\text{enc}(X) || m),
    $$
    Za konec je potrebno izračunati še 
    $$ 
    y = es + r \bmod q, 
    $$
    podpis sporočila $m$ pa je potem par $(X, y)$ oz.\ 
    $$ 
    S(s, m) = (X, y).
    $$
    Postopek je torej enak splošni verziji Schnorrovega podpisa, le da je v tem primeru uporabljeno
    modularno množenje.

    \item \textbf{Preverjanje}:
    Za preverjanje veljavnosti podpisa $(X', y')$ sporočila $m$, je potrebno izračunati 
    $$ 
    e' = H(\text{enc}(X') || m)
    $$
    in preveriti, če velja 
    \begin{equation}
        g^{y'} \stackrel{?}{\equiv} X' \cdot I^{e'} \pmod p. \label{eq:schnorr-ver}
    \end{equation}
\end{itemize}

Z nekaj modularne aritmetike lahko pokažemo, da enačba~\eqref{eq:schnorr-ver} preverja veljavnost
Schnorrovega podpisa.

Po trditvi~\ref{trd:mod-q} lahko levo stran enačbe za preverjanje Schnorrovega podpisa~\eqref{eq:schnorr-ver}
prepišemo
\begin{align*}
    g^{y'} \bmod p &= g^{es + r \bmod q} \bmod p = \\ 
                   &= g^{es + r} \bmod p. 
\end{align*}
Desno stran enačbe~\eqref{eq:schnorr-ver} pa po trditvi~\ref{trd:mod-mn-pt} lahko prepišemo
\begin{align*}
X' \cdot I^{e'} \bmod p &= g^r \bmod p \cdot (g^s \bmod p)^e \bmod p = \\
                        &\stackrel{\eqref{eq:mod-exp}}{=} (g^r \bmod p) \cdot (g^{es} \bmod p) \bmod p = \\ 
                        &\stackrel{\eqref{eq:mod-prod}}{=} g^{es + r} \bmod p,
\end{align*}
kjer smo pri prehodu v drugo vrstico uporabili lastnost~\eqref{eq:mod-exp}, pri prehodu v tretjo 
pa lastnost~\eqref{eq:mod-prod}. Ker se obe strani ujemata za veljavne podpisne vrednosti, ta enačba 
res preverja Schnorrov podpis.
