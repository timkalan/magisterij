% LTeX: language=sl-SI
\section{Zaključek}
% intro
V sodobnem digitalnem svetu, kjer sta varnost in avtentičnost podatkov ključnega pomena, digitalni
podpisi predstavljajo temeljni gradnik varne komunikacije. Ta magistrska naloga se je osredotočila
na področje večstranskih digitalnih podpisov, ki omogočajo skupini posameznikov, da skupaj podpišejo
sporočilo z enim samim, kratkim podpisom. Takšni podpisi so še posebej relevantni v kontekstu
tehnologije veriženja blokov, kjer je učinkovitost transakcij ključnega pomena.

% povzetek
Naloga se prične s celovitim pregledom kriptografskih osnov, nujnih za razumevanje digitalnih podpisov,
vključno z modularno aritmetiko, zgoščevalnimi funkcijami in kriptografijo javnega ključa. Prvi del
naloge je posvečen Schnorrovemu podpisu, ki se je uveljavil kot standarden gradnik za večstranske
podpise.

V nadaljevanju smo predstavili in analizirali formalni model za večstranske podpise, znan kot večstranski
podpis podskupine z odgovornostjo (ASM), ki so ga prvi definirali Micali, Ohta in Reyzin~\cite{micali2001asm}.
Ta model zagotavlja, da je iz podpisa mogoče razbrati, kateri člani so sodelovali pri njegovem
ustvarjanju, kar omogoča polno odgovornost in prilagodljivost. Za zagotavljanje varnosti pred napadi,
kot je napad na generiranje ključev, shema uporablja dokaze znanja brez razkritja znanja in Merklova
drevesa. Varnost te sheme smo dokazali v modelu slučajnega oraklja, pri čemer smo se oprli na lemo o
razcepu in težavnost problema diskretnega logaritma.

Na koncu smo se osredotočili na sodobnejše pristope in predstavili podpis MuSig2~\cite{jonas2020musig2},
ki je bil razvit kot odgovor na potrebo po bolj učinkovitih shemah z manj komunikacijskimi krogi.
MuSig2 zmanjša število krogov komunikacije na dva, kar predstavlja znatno izboljšavo v primerjavi s
tremi krogi, ki jih zahteva klasični večstranski Schnorrov podpis. Cena za to učinkovitost je opustitev
lastnosti odgovornosti in prilagodljivosti, saj MuSig2 ne razkriva identitete podpisnikov, kar pa je
v določenih aplikacijah, kot je Bitcoin~\cite{nakamoto2009bitcoin}, zaželena lastnost za večjo zasebnost.

% sklepne misli
Empirična analiza, izvedena v okviru naloge, je potrdila teoretične prednosti in slabosti obravnavanih
shem. Medtem ko čas preverjanja pri navadnem Schnorrovem podpisu linearno narašča s številom
podpisnikov, je čas preverjanja pri večstranskem Schnorrovem podpisu bistveno krajši, pri MuSig2 pa
je celo konstanten in neodvisen od števila podpisnikov. To dejstvo daje večstranskim podpisom veliko
prednost pri aplikacijah, kjer je delo preverjevalca omejeno ali drago. Po drugi strani pa večstranski
podpisi zahtevajo komunikacijo med podpisniki, kar lahko predstavlja oviro v primerih, kjer je omejeno
delo podpisnikov. 
Večstranski Schnorrov podpis zahteva tri kroge komunikacije in kompleksnejše generiranje ključev,
medtem ko MuSig2 potrebuje le dva kroga. Izbira med shemami je tako odvisna od specifičnih potreb
aplikacije: kjer sta ključni prilagodljivost in odgovornost, je večstranski Schnorrov podpis kljub
dodatni komunikaciji prava izbira. Kjer pa so v ospredju učinkovitost preverjanja, zasebnost in
enostavna integracija, ima MuSig2 veliko prednost.
% Podpis MuSig2 predstavlja pomemben korak v razvoju večstranskih podpisov. V primerih, ko želimo, da
% se celotna skupina dokazljivo strinja s sporočilom, a ne želimo razkriti članov, je MuSig2 odlična
% izbira. Poleg učinkovitosti in varnosti, je za praktično uporabo pomembno tudi, da vrača Schnorrove
% podpise.
%
% Če pa želimo podpisno shemo, ki omogoča popolno prilagodljivost in odgovornost, pa je (vsaj zaenkrat)
% potrebno uporabiti tri kroge komunikacije in podpis kot je recimo večstranski Schnorrov podpis. Čeprav
% ni najbolj učinkovit, je predstavljal prvi korak k razvoju MuSig2, saj so njegovi avtorji prvi, ki so
% predstavili formalni model in dokazali varnost večstranskih podpisov.

% nadaljnje delo
Nadaljnje delo bi se lahko osredotočilo na primerjavo predstavljenih shem s post-kvantnimi shemami,
ki so zasnovane za zaščito pred potencialnimi napadi kvantnih računalnikov. Tu se sicer pojavijo
velike razlike med shemami, saj post-kvantne sheme ne temeljijo na problemu diskretnega logaritma.
V svetu večstranskih podpisov še ne obstaja standardna post-kvantna shema, veliko kandidatov pa temelji
na problemih na rešetkah (ang.\ \textit{lattice problems}), kot je na primer MuSig-L~\cite{boschini2022musigl}.
Ta shema je še posebaj zanimiva za nas, saj je njen cilj biti post-kvantna različica MuSig2.

Možna nadgradnja empirične analize bi lahko bila natančnejša analiza učinkovitosti z upoštevanjem realnih
pogojev komunikacijskih omrežij, saj je bila v tej nalogi komunikacija zanemarjena. Podpise bi lahko
implementirali v realnem sistemu (npr.\ na izbrani verigi blokov) in izvedli realne meritve.
Prav tako bi bilo smiselno raziskati optimizacije, kot je shranjevanje Merklovega drevesa med
zaporednimi podpisi, kar bi še dodatno povečalo učinkovitost večstranskega Schnorrovega podpisa.
